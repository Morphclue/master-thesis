\section{Diskussion}
In diesem Kapitel werden die Ergebnisse des Experiments diskutiert.
Dabei werden die theoretischen Synergien aus \autoref{subsec:synergies} mit den Ergebnissen verglichen.
Zusätzlich werden die Ergebnisse mit Erkenntnissen aus der Literatur verglichen und weitere Forschungsmöglichkeiten aufgezeigt.
Es werden in diesem Kontext auch Einschränkungen hinsichtlich des Experiments und der Evaluation diskutiert.\\

Die theoretischen Konzepte konnten in der Evaluation teils bestätigt werden.
Es ist während der Interviews aufgefallen, dass Testpersonen sich aktiv Gedanken über ihre aktuelle Perspektive auf die Inhalte machen.
Zusätzlich dazu machen sie sich Gedanken über andere Filterblasen und begeben sich aktiv in diese.
Die Darstellung scheint somit tatsächlich die Neugier von Nutzenden durch die Darstellung zu wecken.
Bei den Interviews ist besonders die Übersichtlichkeit des Begriffsverbandes aufgefallen. \\

Das Konzept vom Pull wurde jedoch nicht von allen Testpersonen positiv wahrgenommen.
Der Grund dafür liegt darin, dass das aktive Suchen von Inhalten als wesentlich anstrengender empfunden wird.
Dieser Punkt sollte in weiteren Studien näher untersucht werden.
Dabei wären mehrere Aspekte von hoher Relevanz.
Zum einen sollte die Anstrengung des aktiven Suchens von Inhalten möglichst gering gehalten werden, damit die Arbeitsbelastung nicht zu hoch ist.
Zum anderen sollte der Faktor Zeit viel stärker berücksichtigt werden.
Dabei sollten unterschiedliche Kontexte untersucht werden und die Zeit, die für das Finden von Inhalten benötigt wird, gemessen werden.
Ein wichtiger Kontext ist dabei die mobile Nutzung, welcher bei dieser Arbeit zunächst bewusst ausgeschlossen wurde.
Die mobile Nutzung ist jedoch ein wichtiger Aspekt, da Nutzende in diesem Kontext in kurzen Intervallen Inhalte konsumieren.
Aus diesem Grund ist es wichtig, dass interessante und sehenswerte Inhalte schnell gefunden werden können.
Zusätzlich ist es aktuell unklar inwieweit die Navigation durch den Begriffsverband in der mobilen Nutzung gut funktioniert.\\

Bezüglich der Inhalte lässt sich ebenfalls sagen, dass es interessant wäre, weitere Strukturen für die Darstellung von Inhalten zu untersuchen.
Dabei wäre vor allem die Darstellung einer großen Datenmenge interessant, da die Darstellung von großen Datenmengen andere Herausforderungen mit sich bringt.
Interessant könnte hierbei die Verwendung von Teilverbänden sein, welche eine Art Zoom-In und Zoom-Out in Themengebiete ermöglichen.
Zusätzlich stellt sich die Frage inwieweit sehenswerte Inhalte hervorgehoben werden sollten.
Wichtig könnten besonders politische Ereignisse sein, da diese eine hohe Relevanz für die Demokratie haben.
Dabei treten ebenfalls ethische Fragen auf, da die Hervorhebung von Inhalten eine Beeinflussung der Nutzenden darstellt.
Dieser Aspekt sollte in weiteren Studien unter Berücksichtigung an die unterschiedlichen Modelle der Demokratie untersucht werden.\\

%%% Einschränkungen
%% Experiment 
% Datensatz
% Fragestellungen
% Stichprobe

%% Begriffsverband
% RS riesige Datenmengen
% Begriffsverband nicht allzu gut skalierbar
% Wiederum Liste



% Weitere Forschung