\section{Diskussion}\label{sec:discussion}
In diesem Kapitel werden die Ergebnisse des Experiments diskutiert.
Dabei werden die theoretischen Synergien aus \autoref{subsec:synergies} mit den Ergebnissen verglichen.
Zusätzlich werden die Ergebnisse mit Erkenntnissen aus der Literatur verglichen und weitere Forschungsmöglichkeiten aufgezeigt.
Es werden in diesem Kontext auch Einschränkungen hinsichtlich des Experiments und der Evaluation diskutiert.\\

% Theoretischen Konzepte
Die theoretischen Konzepte konnten in der Evaluation teils bestätigt werden.
Es ist während der Interviews aufgefallen, dass Testpersonen sich aktiv Gedanken über ihre aktuelle Perspektive auf die Inhalte machen.
Zusätzlich dazu machen sie sich Gedanken über andere Filterblasen und begeben sich aktiv in diese.
Der Begriffsverband scheint somit tatsächlich die Neugier von Nutzenden durch die Darstellung zu wecken.
Bei den Interviews ist besonders die Übersichtlichkeit des Begriffsverbandes aufgefallen. \\

% Welche Kategorie?
Besonders die explizite Erwähnung einer Testperson \glqq In welche Kategorie will ich mich denn jetzt begeben?\grqq{} zeigt, dass das Experiment auf den ersten Blick erfolgreich war.
Dies zeigt, dass ein Umdenken stattgefunden hat.
Schließlich war das Ziel des Experiments, dass Personen sich zuerst Gedanken über die Kategorie machen und sich im Anschluss Artikel aus dieser Kategorie aussuchen.
Dies müsste jedoch noch weiter verfolgt werden, um eine konkretere Aussage treffen zu können.
Denn es ist nicht klar, ob dies auch eine Auseinandersetzung mit kritischeren Inhalten fördert.

% Pull & Time
Das Konzept vom Pull wurde jedoch nicht von allen Testpersonen positiv wahrgenommen.
Der Grund dafür liegt darin, dass das aktive Suchen von Inhalten als wesentlich anstrengender empfunden wird.
Dieser Punkt sollte in weiteren Studien näher untersucht werden.
Dabei wären mehrere Aspekte von hoher Relevanz.
Zum einen sollte die Anstrengung des aktiven Suchens von Inhalten möglichst gering gehalten werden, damit die Arbeitsbelastung nicht zu hoch ist.
Zum anderen sollte der Faktor Zeit viel stärker berücksichtigt werden.
Dabei sollten unterschiedliche Kontexte untersucht werden und die Zeit, die für das Finden von Inhalten benötigt wird, gemessen werden.
Ein wichtiger Kontext ist dabei die mobile Nutzung, welcher bei dieser Arbeit zunächst bewusst ausgeschlossen wurde.
Die mobile Nutzung ist jedoch ein wichtiger Aspekt, da Nutzende in diesem Kontext in kurzen Intervallen Inhalte konsumieren.
Aus diesem Grund ist es wichtig, dass interessante und sehenswerte Inhalte schnell gefunden werden können.
Zusätzlich ist es aktuell unklar inwieweit die Navigation durch den Begriffsverband in der mobilen Nutzung gut funktioniert.\\

% Weitere Strukturen
Bezüglich der Inhalte lässt sich ebenfalls sagen, dass es interessant wäre, weitere Strukturen für die Darstellung von Inhalten zu untersuchen.
Dabei wäre vor allem die Darstellung einer großen Datenmenge interessant, da die Darstellung von großen Datenmengen ebenfalls Herausforderungen mit sich bringt.
Interessant könnte hierbei die Verwendung von Teilverbänden sein, welche eine Art Zoom-In und Zoom-Out in Themengebiete ermöglichen.
Primär müsste eine Lösung geschafft werden, welche die aktuelle Listendarstellung im Begriffsverband ersetzt.
Der Grund hierfür ist, dass die Darstellung als Liste im Begriffsverband die gleichen Probleme aufweist, wie die Darstellung als Liste auf Nachrichtenportalen.
Zusätzlich stellt sich die Frage inwieweit sehenswerte Inhalte hervorgehoben werden sollten.
Wichtig könnten besonders politische Ereignisse sein, da diese eine hohe Relevanz für die Demokratie haben.
Dabei treten ebenfalls ethische Fragen auf, da die Hervorhebung von Inhalten eine Beeinflussung der Nutzenden darstellt.
Dieser Aspekt sollte in weiteren Studien unter Berücksichtigung an die unterschiedlichen Modelle der Demokratie untersucht werden.\\

% Politische Debatte
Die Masterarbeit beleuchtet zudem nur einen Teil der wichtigen Aspekte einer Demokratie.
Zwar ist es möglich sich mithilfe des Begriffsverbands zu informieren, jedoch wurde die aktive Teilnahme an der politischen Debatte nicht näher untersucht.
Laut Dewey ist die aktive Teilnahme an der politischen Debatte jedoch Voraussetzung für eine funktionierende Demokratie.
Da Kommentarspalten auf Nachrichtenportal oftmals ebenfalls in einer Liste dargestellt werden, könnte es interessant sein diese ebenfalls umzustrukturieren.
Durch neue Darstellungsformen könnte es möglich sein, Nutzende zu motivieren sich aktiver an der politischen Debatte zu beteiligen.
Ebenfalls könnte dadurch ein übersichtlicheres Meinungsbild entstehen, welches die politische Debatte bereichern könnte.
Die Darstellung als Begriffsverband könnte ebenfalls verwendet werden, um Kommentare zu strukturieren.
Der Vorteil dadurch wäre, dass Nutzende sich bewusst werden aus welcher Perspektive oder Rechtfertigung heraus ein Kommentar geschrieben wurde.
Dadurch könnte es möglich sein, dass Nutzende eine geringere Abwehrhaltung gegenüber anderen Meinung bilden.
Ebenfalls wäre es vorstellbar, dass Nutzende sich mehr für andere Perspektiven interessieren.
Diese Darstellung könnte auf der anderen Seite aber auch mehr Möglichkeiten für gezielte Manipulation oder Hassangriffe bieten.
Daher sollte diese und weitere Darstellungsformen in weiteren Studien untersucht werden.\\

% Kategorisierung Kommentare
Ebenfalls stellt sich in diesem Zusammenhang die Frage, wie die Kommentare kategorisiert werden sollen.
Dies betrifft nicht nur die Kommentare, sondern alle Artikel, welche im Begriffsverband dargestellt werden sollen.
In dieser Arbeit wurde die Kategorisierung von Inhalten manuell von Fachpersonen durchgeführt.
Dies ist jedoch nicht skalierbar und somit nicht für eine praktische Anwendung geeignet.
Jedoch ist auch die automatische Kategorisierung von Inhalten nicht trivial.
Obwohl Technologien wie \ac{NLP} bereits Fortschritte bei der Kategorisierung von Inhalten ermöglicht haben, existieren viele Herausforderungen, welche zunächst bewältigt werden müssen \cite{nlp}.
\ac{NLP} bezieht sich auf die Verarbeitung von natürlicher Sprache durch Computer.
Dabei werden Techniken aus der Linguistik und Künstlichen Intelligenz verwendet.
Beispiele für Herausforderungen von \ac{NLP} sind die Erkennung von Ironie, Ambiguität oder Kontext in Texten.
An dieser Stelle sei angemerkt, dass eine Studie existiert, welche die automatische Kategorisierung von Inhalten in die Welten der \ac{EC} untersucht hat \cite{solans}.
Jedoch war auch bei dieser Studie die manuelle Kategorisierung von Inhalten notwendig. \\

% Stichprobe
Für weitere Studien ist es von großer Notwendigkeit das Experiment zu erweitern und auszubessern.
Zunächst sollte die Anzahl der Testpersonen erhöht werden.
Das Experiment sollte dabei so aufgebaut werden, dass die Stichprobe repräsentativ für die Bevölkerung in Deutschland ist.
In weiteren Studien könnte diese Stichprobe auch auf andere Länder ausgeweitet werden.
Dabei sollten die Regulierungen und Werte der jeweiligen Länder berücksichtigt werden.
Zusätzlich könnten bei einer größeren Stichprobe die Reihenfolge der Prototypen zufällig ausgewählt werden.
Auch der Datensatz sollte wesentlich größer sein, um mehr Möglichkeiten für die Navigation durch Artikel zu bieten.
Dies würde zufällige Ergebnisse reduzieren und die Aussagekraft des Diversitätsmaßes erhöhen.
Sinnvoll könnte es ebenfalls sein, den gleichen Datensatz für beide Prototypen zu verwenden.
Dies würde die Untersuchung vereinfachen und die Ergebnisse vergleichbarer machen.\\

% Aufgabenstellung
Das Tutorial und die mit den Prototypen verbundenen Aufgabenstellungen müssten ebenfalls weiterentwickelt werden.
Dabei sollte der Fokus viel stärker auf die Darstellung und Navigation gelegt werden, sodass Testpersonen nicht dazu verleitet werden nur auf die Überschriften von Artikeln zu achten.
Es sollte zusätzlich in Betracht gezogen werden nicht beide Prototypen gleichzeitig an einer Person zu testen.
Der Grund dafür ist, dass das Experiment bereits in dieser Form sehr zeitaufwändig ist.
Im Schnitt dauert das Experiment eine Stunde und dies führt zu einer hohen Arbeitsbelastung für die Testpersonen.
Daher sollte in weiteren Studien darauf geachtet werden, dass die Arbeitsbelastung möglichst gering gehalten wird.
Eine weitere Möglichkeit wäre, dass die Testpersonen die Prototypen zeitversetzt testen.
Dies würde die Arbeitsbelastung reduzieren und die Ergebnisse pro Testperson vergleichbarer machen.\\
