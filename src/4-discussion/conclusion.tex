\section{Fazit}\label{sec:conclusion}
In dieser Arbeit wurde eine Darstellung von News-Artikeln in einem Begriffsverband entwickelt und evaluiert.
Dabei kann die Forschungsfrage, ob die Darstellung von Inhalten in einem Begriffsverband die Auseinandersetzung mit unterschiedlichen Perspektiven fördert, mit Ja beantwortet werden.
Zwar existieren einige Einschränkungen, jedoch konnte an verschiedenen Stellen gezeigt werden, dass das Experiment erfolgreich war.
Durch den Fragebogen AttrakDiff wurde deutlich, dass auch das persönliche Wachstum durch den Begriffsverband gefördert wird.
Aussagen der Testpersonen in den Interviews zufolge konnte zusätzlich gezeigt werden, dass Neugier an anderen Perspektiven geweckt wird.
Dies ist in einer traditionellen Darstellung von Artikeln als Liste nicht möglich, da diese bereits aufgrund des Designs auf wenig Diversität ausgelegt sind.\\

Weiterhin lässt sich festhalten, dass das Tutorial einen positiven Einfluss auf das Verständnis der Darstellung hat.
Zwar weist das Tutorial einige Limitationen auf, jedoch konnte gezeigt werden, dass Testpersonen das Wissen aus dem Tutorial bei der Navigation im Begriffsverband anwenden konnten.
Dies ist ein wichtiger Schritt, um die Darstellung für eine breite Masse zugänglich zu machen.
Schließlich ist es wichtig, dass die Darstellung intuitiv bedienbar ist, damit sie von Nutzenden akzeptiert wird.
Aktuell ist die Darstellung, laut berechneten \ac{QUESI}-Score, noch nicht intuitiv genug, da sie eine zu hohe Arbeitsbelastung für Nutzende darstellt.
Dies ist jedoch nicht verwunderlich, da die Darstellung neuartig ist und sich Nutzende erst an die neue Darstellung gewöhnen müssen.
Trotzdem ist es wichtig, dass die Umgewöhnung möglichst gut unterstützt wird. \\

Dennoch bleibt das Problem der Skalierbarkeit bestehen, was bedeutet, dass weitere Forschung notwendig ist, um die Darstellungsform für eine größere Anzahl an Artikeln zu optimieren.
Ebenfalls ist es notwendig, weitere Experimente durchzuführen, um die Ergebnisse mit einer größeren Stichprobe zu validieren oder neue Darstellungsformen zu erproben.
Insgesamt kann festgehalten werden, dass die Ergebnisse der erprobten Darstellungsform wichtige Erkenntnisse und Einsichten für die Darstellung von News-Artikeln in einem Begriffsverband liefert.
Trotz einiger Einschränkungen konnte gezeigt werden, dass die verwendeten Methoden Potenzial bieten, um die Darstellung hinsichtlich unterschiedlicher Perspektiven zu verbessern.
Durch weitere Forschung und Entwicklung wäre es möglich, dieses Potenzial noch besser auszuschöpfen.
