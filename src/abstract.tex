\newpage

\thispagestyle{empty}

\vspace*{1cm}
\begin{center}
    {\Huge \bf Zusammenfassung}
\end{center}
\vspace*{1.4cm}

Empfehlungssysteme unterstützen Nutzende bei der Informationsbeschaffung im Internet, indem sie basierend auf den Interessen der Nutzenden Inhalte vorschlagen.
Dadurch besteht die Gefahr, dass Nutzende in einer Filterblase gefangen sind und ausschließlich Inhalte erhalten, die ihren eigenen Interessen entsprechen.
Ziel ist die Entwicklung eines alternativen Navigationssystems, welches Nutzende dazu anregt sich mit unterschiedlichen Inhalten auseinanderzusetzen und ihre persönliche Filterblase zu durchbrechen.
Der Fokus liegt dabei auf einer Darstellung von Informationen als Begriffsverband, die sich von herkömmlichen Listendarstellungen im Internet unterscheidet.
Um die Wirkung des Systems zu untersuchen, wurde eine qualitative Studie mit 10 Teilnehmenden durchgeführt.
Die Ergebnisse zeigen, dass die Teilnehmenden durch das System dazu angeregt werden, sich mit vielfältigen Inhalten auseinanderzusetzen.
