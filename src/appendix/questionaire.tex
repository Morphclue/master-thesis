\section{Interview Leitfaden \& Fragebogen}
\textcolor{red}{TODO}

\subsection{Intro}
\textcolor{red}{TODO}
\begin{itemize}
    \item Datenschutz
    \item Gedanken laut aussprechen
    \item Setting einleiten
\end{itemize}

\subsection{Setting}
Du interessiert dich für das Thema E-Mobilität und hast dich dazu entschieden, dich darüber zu informieren.

\subsection{Aufgaben - Listenansicht}
Informiere dich zum Thema E-Mobilität.
Navigiere dich wie von anderen News-Websites gewohnt durch die Artikel.
Die hier dargestellte Listenansicht steht stellvertretend für die bereits bekannte Listen im Internet.
\begin{enumerate}
    \item Finde und klicke den Artikel \glqq Strom, Trassen, Verteilernetze\grqq{}.
    \item Finde und klicke den Artikel \glqq Autobauer als Software-Riesen\grqq{}.
    \item Welcher Artikel ist am ähnlichsten zu \glqq Autobauer als Software-Riesen\grqq{}?
    \item Welcher Artikel unterscheidet sich am meisten von \glqq Autobauer als Software-Riesen\grqq{}?
\end{enumerate}

\subsection{Aufgaben - Formale Begriffsanalyse}
Informiere dich zum Thema E-Mobilität.
Verwende dazu das erworbene Wissen aus dem Tutorial, um dich durch die neue Darstellungsform zu navigieren.
\begin{enumerate}
    \item Finde und klicke den Artikel \glqq Harter Schlag für Hersteller Plug-in Prämie fällt weg\grqq{}.
    \item Finde und klicke den Artikel \glqq Deutlich sauberer als gedacht\grqq{}.
    \item Welcher Artikel ist am ähnlichsten zu \glqq Deutlich sauberer als gedacht\grqq{}?
    \item Welcher Artikel unterscheidet sich am meisten von \glqq Deutlich sauberer als gedacht\grqq{}?
\end{enumerate}

\subsection{Fragebogen}
Wir haben uns jetzt mit beiden Darstellungsformen beschäftigt.
Ich habe noch einen weiteren Fragebogen für dich, welchen du mündlich beantworten kannst.
\begin{itemize}
    \item Welche Darstellungsformen für Online-News-Artikel sind dir bekannt?
    \item Wie häufig besuchst du News-Webseiten?
    \item Inwieweit hast du bereits Vorerfahrung zum Thema E-Mobilität?
    \item Könntest du dir vorstellen einer der beiden gezeigten Darstellungsformen persönlich zu nutzen?
    \item Hast du bereits mit Graphen gearbeitet?
    \item Denkst du, dass die dir gezeigte neue Darstellungsform ein diverses Bild der Elektromobilität zeigt?
    \item Wie beeinflusst das Aussehen einer Webseite deine Entscheidung, welche Informationen du liest und wie lange du auf der Seite bleibst?
\end{itemize}

\subsection{Demografische Daten}
Ich habe deine Daten zwar bereits mithilfe vom AttrakDiff-Fragebogen erfasst, aber dieser ist leider etwas ungenau.
Deswegen bitte ich dich, mir noch folgende Daten kurz mitzuteilen.
\begin{itemize}
    \item Alter
    \item Geschlecht
    \item Bildungsstand
    \item Beruf
\end{itemize}
