\section{Interview Leitfaden \& Fragebogen}
\textcolor{red}{TODO}

\subsection{Intro}
\textcolor{red}{TODO}
\begin{itemize}
    \item Datenschutz
    \item Gedanken laut aussprechen
\end{itemize}

\subsection{Aufgaben}
\textcolor{red}{TODO}
\begin{enumerate}
    \item Informiere dich zum Thema E-Mobilität. Klicke dazu wie von anderen News-Websites gewohnt auf die entsprechenden Artikel.
    \item Finde und klicke den Artikel \glqq Strom, Trassen, Verteilernetze\grqq{}
    \item Finde und klicke den Artikel \glqq Autobauer als Software-Riesen\grqq{}
    \item Welcher Artikel ist am ähnlichsten zu \glqq Autobauer als Software-Riesen\grqq{}
    \item Welcher Artikel unterscheidet sich am meisten von \glqq Autobauer als Software-Riesen\grqq{}
\end{enumerate}

Die Ökonomie der Konventionen beschreibt verschiedene Welten mit diversen Werten, welche die Entscheidungen, Handlungen und Meinungen von Menschen oder Strukturen repräsentieren.
Dabei basieren diese Welten auf der Annahme, dass man diese auf unterschiedliche Arten interpretiert werden können.
Diese Interpretation wird als Rechtfertigung für eine Welt bezeichnet.
Im Kontext des Experiments geht es um Rechtfertigungen im Online-Journalismus.
Diese können sich positiv, negativ oder beides bezüglich der Werte einer Welt äußern.
Die konkreten Werte einer Welt musst du nicht im Detail verstehen.
Es ist nicht notwendig, die konkreten Werte einer Welt im Detail zu verstehen, es reicht aus, die Welten als eine Art Kategorisierung der Artikel zu betrachten.

\begin{enumerate}
    \item Informiere dich zum Thema E-Mobilität. Verwende dazu das erworbene Wissen aus dem Tutorial, um dich durch die neue Darstellungsform zu navigieren.
    \item Finde und klicke den Artikel \glqq Harter Schlag für Hersteller Plug-in Prämie fällt weg\grqq{}
    \item Finde und klicke den Artikel \glqq Deutlich sauberer als gedacht\grqq{}
    \item Welcher Artikel ist am ähnlichsten zu \glqq Deutlich sauberer als gedacht\grqq{}
    \item Welcher Artikel unterscheidet sich am meisten von \glqq Deutlich sauberer als gedacht\grqq{}
\end{enumerate}

\subsection{Fragebogen}
\textcolor{red}{TODO}

\begin{itemize}
    \item Welche Darstellungsformen für Online-News-Artikel sind Ihnen bekannt?
    \item Wie häufig besuchen Sie News-Webseiten? (x pro Woche?)
    \item Inwieweit hast du bereits Vorerfahrung zum Thema E-Mobilität?
    \item Könnten Sie sich vorstellen einer der beiden gezeigten Darstellungsformen persönlich zu nutzen?
    \item Haben Sie bereits mit Graphen gearbeitet?
    \item Wie bewerten Sie die Diversität der Perspektiven, die Sie durch das (alternative) Navigationsparadigma erhalten?
    \item Wie beeinflusst die Art und Weise, wie Inhalte präsentiert werden, Ihre Informationsbeschaffung und Informationskonsum?
\end{itemize}

\subsection{Demografische Daten}
\textcolor{red}{TODO}
\begin{itemize}
    \item Alter
    \item Geschlecht
    \item Bildungsstand
    \item Beruf
\end{itemize}