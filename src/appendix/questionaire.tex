\section{Interview Leitfaden \& Fragebogen}
\textcolor{red}{TODO}

\subsection{Intro}
\textcolor{red}{TODO}
\begin{itemize}
    \item Datenschutz
    \item Gedanken laut aussprechen
    \item Setting einleiten
\end{itemize}

\subsection{Setting}
Du interessiert dich für das Thema E-Mobilität und hast dich dazu entschieden, dich darüber zu informieren.

\subsection{Aufgaben - Listenansicht}
Informiere dich zum Thema E-Mobilität. Navigiere dich wie von anderen News-Websites gewohnt durch die Artikel.
\begin{enumerate}
    \item Finde und klicke den Artikel \glqq Strom, Trassen, Verteilernetze\grqq{}
    \item Finde und klicke den Artikel \glqq Autobauer als Software-Riesen\grqq{}
    \item Welcher Artikel ist am ähnlichsten zu \glqq Autobauer als Software-Riesen\grqq{}
    \item Welcher Artikel unterscheidet sich am meisten von \glqq Autobauer als Software-Riesen\grqq{}
\end{enumerate}

\subsection{Aufgaben - Formale Begriffsanalyse}
Die Kategorisierung erfolgt basierend auf unterschiedlichen Welten, welche du bereits im Tutorial kennengelernt hast.
Sie beschreibt verschiedene Welten mit diversen Werten, welche die Entscheidungen, Handlungen und Meinungen von Menschen oder Strukturen repräsentieren.
Dabei basieren diese Welten auf der Annahme, dass man diese auf unterschiedliche Arten interpretiert werden können.
Diese Interpretation wird als Rechtfertigung für eine Welt bezeichnet.
Im Kontext des Experiments geht es um Rechtfertigungen im Online-Journalismus.
Diese können sich positiv, negativ oder beides bezüglich der Werte einer Welt äußern.
Es ist nicht notwendig, die konkreten Werte einer Welt im Detail zu verstehen, es reicht aus, die Welten als eine Art Kategorisierung der Artikel zu betrachten.
Informiere dich zum Thema E-Mobilität. Verwende dazu das erworbene Wissen aus dem Tutorial, um dich durch die neue Darstellungsform zu navigieren.

\begin{enumerate}
    \item Finde und klicke den Artikel \glqq Harter Schlag für Hersteller Plug-in Prämie fällt weg\grqq{}
    \item Finde und klicke den Artikel \glqq Deutlich sauberer als gedacht\grqq{}
    \item Welcher Artikel ist am ähnlichsten zu \glqq Deutlich sauberer als gedacht\grqq{}
    \item Welcher Artikel unterscheidet sich am meisten von \glqq Deutlich sauberer als gedacht\grqq{}
\end{enumerate}

\subsection{Fragebogen}
\textcolor{red}{TODO}

\begin{itemize}
    \item Welche Darstellungsformen für Online-News-Artikel sind dir bekannt?
    \item Wie häufig besuchst du News-Webseiten? (x pro Woche?)
    \item Inwieweit hast du bereits Vorerfahrung zum Thema E-Mobilität?
    \item Könntest du dir vorstellen einer der beiden gezeigten Darstellungsformen persönlich zu nutzen?
    \item Hast du bereits mit Graphen gearbeitet?
    \item Denkst du, dass die dir gezeigte neue Darstellungsform ein diverses Bild der Elektromobilität zeigt?
    \item Wie beeinflusst das Aussehen einer Webseite deine Entscheidung, welche Informationen du liest und wie lange du auf der Seite bleibst?
\end{itemize}

\subsection{Demografische Daten}
\textcolor{red}{TODO}
\begin{itemize}
    \item Alter
    \item Geschlecht
    \item Bildungsstand
    \item Beruf
\end{itemize}