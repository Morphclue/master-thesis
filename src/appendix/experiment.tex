\section{Interview Leitfaden}
Dieser Leitfaden dient als Vorlage für das Interview mit den Testpersonen.
Dabei ist die Reihenfolge der Aufgaben abhängig von der Gruppe, in der sich die Testperson befindet.

\subsection{Intro}
Bevor ich das Experiment starte, möchte ich dir kurz den Ablauf und einige wichtige Informationen erklären. 
Zuerst werde ich sicherstellen, dass du über den Datenschutz im Rahmen dieses Experiments informiert bist.
Alle Daten, die ich erfasse, werden anonymisiert und vertraulich behandelt.
Nachdem das Experiment abgeschlossen und die Daten transkribiert wurden, werden diese gelöscht.
Während des Experiments möchte ich dich bitten, deine Gedanken laut auszusprechen.
Das Setting, in dem du dich befindest, lautet wie folgt: \glqq Du interessierst dich für das Thema Elektromobilität und hast dich dazu entschieden, dich darüber zu informieren.\grqq{}

\subsection{Aufgaben - Listenansicht}
Informiere dich zum Thema E-Mobilität.
Navigiere dich wie von anderen News-Websites gewohnt durch die Artikel.
Die hier dargestellte Listenansicht steht stellvertretend für die bereits bekannte Listen im Internet.
\begin{enumerate}
    \item Finde und klicke den Artikel \glqq Strom, Trassen, Verteilernetze\grqq{}.
    \item Finde und klicke den Artikel \glqq Autobauer als Software-Riesen\grqq{}.
    \item Welcher Artikel ist am ähnlichsten zu \glqq Autobauer als Software-Riesen\grqq{}?
    \item Welcher Artikel unterscheidet sich am meisten von \glqq Autobauer als Software-Riesen\grqq{}?
\end{enumerate}

\subsection{Aufgaben - Formale Begriffsanalyse}
Informiere dich zum Thema E-Mobilität.
Verwende dazu das erworbene Wissen aus dem Tutorial, um dich durch die neue Darstellungsform zu navigieren.
\begin{enumerate}
    \item Finde und klicke den Artikel \glqq Harter Schlag für Hersteller Plug-in Prämie fällt weg\grqq{}.
    \item Finde und klicke den Artikel \glqq Deutlich sauberer als gedacht\grqq{}.
    \item Welcher Artikel ist am ähnlichsten zu \glqq Deutlich sauberer als gedacht\grqq{}?
    \item Welcher Artikel unterscheidet sich am meisten von \glqq Deutlich sauberer als gedacht\grqq{}?
\end{enumerate}