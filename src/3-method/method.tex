\section{Methodik}
% Einleitung
In diesem Abschnitt wird das Design des Experiments vorgestellt, sowie eine kurze Beschreibung der verwendeten Technologien und Daten.
Es wird im Detail erläutert wie das Experiment aufgebaut ist und welche Ziele damit verfolgt werden.
Die konkrete Auswertung der erhobenen Daten wird im folgenden \autoref{sec:results} vorgestellt.\\

% Datensatz
Für das Experiment liegen die von \textcolor{red}{NAME} bereitgestellten Daten vor.
Es handelt sich dabei um Artikel vom Focus-Magazin\footnote{\url{https://www.focus.de}} zum Thema Elektromobilität.
Diese Artikel wurden durch \textcolor{red}{Soziologen} aufbereitet und in die, im \autoref{subsec:economics-of-conventions} erwähnten, Welten der \ac{EC} eingeteilt.
Der Datensatz besteht aus 26 Artikeln, welche in die Welten Markt, Industrie, \textcolor{red}{TODO} und Grün unterteilt wurden.
Im Datensatz ist die Welt als \textcolor{red}{TODO} benannt, weil es in der Literatur unterschiedliche Übersetzungen und Bezeichnungen für die Welten gibt.
Zusätzlich zu der Kategorisierung in eine Welt enthält der Datensatz die Rechtfertigung bezüglich der Welt.
Diese Rechtfertigung gibt an, ob der Artikel für, gegen oder beides bezüglich einer Welt ist.
Dabei bedeutet beides nicht, dass der Artikel neutral ist, sondern dass Textabschnitte im Artikel auftauchen, welche für und gegen die Welt sprechen.\\

% Vorstudie
Es wurde ebenfalls eine qualitative Vorstudie durchgeführt, welches die Grundlage für die Masterarbeit bildet.
Diese Vorstudie wurde durch \textcolor{red}{NAME} durchgeführt und \textcolor{red}{TODO: Anhang + cite Studie}.
Zweck dieser Vorstudie war es neue Darstellungsformen der Artikel-Navigation im Online-Journalismus zu erforschen.
Unter anderem hat sich in dieser Vorstudie herausgestellt, dass die Darstellung als Begriffsverband für fünf von acht Testpersonen als informativ und interessant empfunden wurde.
Allerdings gab es auch zahlreiche Kritikpunkte am Begriffsverband in der Vorstudie.
Der Begriffsverband wies für drei Personen eine hohe Komplexität auf.
Diese sorgt dafür, dass entweder eine lange Einarbeitungszeit nötig ist oder Nutzende abgeschreckt werden.
Ebenfalls ist die Unterteilung in unterschiedliche Rechtfertigung (positiv, negativ oder beides) bezüglich einer Welt für zwei Personen eher verwirrend als sinnvoll.\\

\subsection{Prototyp - Begriffsverband}
Im Rahmen der Masterarbeit wurde ein Prototyp entwickelt, welcher die Ergebnisse der Vorstudie aufgreift und weiterentwickelt.
Er ist somit eine weitere Iteration auf dem Weg zur finalen Umsetzung. 
Dieser Abschnitt erläutert möglichst detailliert die einzelnen Schritte, welche zum Prototypen geführt haben.\\

\begin{figure}[!ht]
    \centering
    \begin{tikzpicture}
        % Erster Graph
        \begin{scope}[every node/.style={circle,thick,draw}]
            \node (A) at (0,1) {};
            \node (B) at (-2,-1) {};
            \node (C) at (2,-1) {};
            \node (D) at (0, -3) {};
        \end{scope}

        % Merkmale
        \draw (A) +(0,0.25) node[above] {Industrie};
        \draw (B) +(0,0.25) node[above, align=left] {Industrie \texttt{+}};
        \draw (C) +(0,0.25) node[above, align=left] {Industrie -};
        \draw (D) +(0,0.25) node[above, align=left] {};

        % Relationen
        \draw (A) -- (B);
        \draw (A) -- (C);
        \draw (B) -- (D);
        \draw (C) -- (D);

        % Zweiter Graph
        \begin{scope}[every node/.style={circle,thick,draw}, xshift=5cm]
            \node (A2) at (2,-1) {};
        \end{scope}

        % Merkmale
        \draw (A2) +(0,0.25) node[above] {Industrie};

        % Unsichtbare Koordinaten für den Pfeil
        \coordinate[right=1.5cm of C] (C_right);
        \coordinate[left=1.5cm of A2] (A2_left);

        % Pfeil zwischen den Graphen
        \draw[-{Stealth[length=3mm]}, thick] (C_right) -- (A2_left);
    \end{tikzpicture}
    \caption{Begriffsverband - Vorstudie (links) und vereinfachte Darstellung (rechts)}
    \label{fig:industry-comparison}
\end{figure}

% Graph
Die hohe Komplexität wurde dadurch reduziert, dass die Anzahl der Elemente im Begriffsverband reduziert wurden.
Da für zwei Personen die Unterteilung in positiv, negativ oder beides verwirrend war, wurden diese Knoten aus dem Begriffsverband entfernt.
Dies wird in \autoref{fig:industry-comparison} beispielhaft dargestellt.
Der Knoten Industrie besitzt in der Vorstudie zwei weitere Nachfolgeknoten, welche eine positive und negative Rechtfertigung für die Welt Industrie darstellen.
Der unterste Knoten vereinigt die Eigenschaften aller darüberliegenden Knoten.
Auf diese Weise gehören Artikel, welche im untersten Knoten liegen zu der Welt Industrie und rechtfertigen sich positiv und negativ bezüglich dieser Welt.\\

\begin{figure}[!ht]
    \centering
    \begin{tikzpicture}
        \definecolor{nodegreen}{HTML}{00A86B}
        \definecolor{nodeyellow}{HTML}{FFB50F}
        \definecolor{nodered}{HTML}{D32F2F}
        \begin{scope}[every node/.style={circle,thick,draw}]
            \clip (-1.1,0) rectangle (-0.34,1.1);
            \node[minimum size=2cm, fill=nodegreen] (A) at (0,0) {};
        \end{scope}
        \begin{scope}[every node/.style={circle,thick,draw}]
            \clip (-0.34,0) rectangle (0.32,1.1);
            \node[minimum size=2cm, fill=nodeyellow] (B) at (0,0) {};
        \end{scope}
        \begin{scope}[every node/.style={circle,thick,draw}]
            \clip (0.32,0) rectangle (1.1,1.1);
            \node[minimum size=2cm, fill=nodered] (C) at (0,0) {};
        \end{scope}

        \coordinate[right=0cm of A] (A_right);
        \coordinate[left=0cm of A] (A_left);
        \draw[-, thick] (A_left) -- (A_right);
    \end{tikzpicture}
    \caption{Einfärbung des Knotens Industrie}
    \label{fig:industry-colored}
\end{figure}

% Knotenfärbung
Der aus der Reduktion resultierende Graph enthält statt der vier Knoten nur noch einen Knoten.
Wenn man dies für den ganzen Graphen anwendet, dann kann der Graph \textcolor{red}{um x Knoten - nochmal nachprüfen} reduziert werden.
Der reduzierte Graph enthält jedoch auch weniger Informationen, da die Rechtfertigungen für die unterschiedlichen Welten nicht mehr dargestellt werden.
Um diesem Problem entgegenzuwirken, wurde die Repräsentation des Knotens aus \autoref{fig:industry-comparison} um drei Farben erweitert.
Diese Farben werden in \autoref{fig:industry-colored} dargestellt.
Die Farben sollen angeben, ob die Artikel sich innerhalb dieses Knotens positiv (grün), negativ (rot) oder neutral (gelb) rechtfertigen.\\

% Drei Farben
Diese drei Farben wurden ausgewählt, weil sie für die meisten Menschen auf dieselbe Weise emotional assoziiert werden.
So zeigen die Daten einer Studie, dass, aus der Menge der Grundfarbtöne, die Farbe Grün am positivsten und Rot am negativsten assoziiert wird \cite{color-emotion}.
Ebenfalls sind Grün und Rot Komplementärfarben, welche zwei Gegenpole darstellen sollen.
Die Farbe Gelb wird in der Studie als größtenteils positiv von Testpersonen wahrgenommen.
Gelb wurde ausgewählt, weil die Farbe Gelb bei Ampeln zum Beispiel als Zwischenzustand zwischen Rot und Grün dient.
Ebenfalls liegt Gelb im Farbspektrum zwischen Rot und Grün und ist ebenfalls ein Grundfarbton.
Auf diese Art und Weise kann möglicherweise eine Verknüpfung im Hirn für die Farbe Gelb als Zwischenzustand hergestellt werden.
An dieser Stelle sei angemerkt, dass Farben je nach Kontext unterschiedlich interpretiert werden können.
So kann die Farbe Rot, laut Studie, als warme und leidenschaftliche, aber auch als aggressive und intensive Farbe wahrgenommen werden.
Die Farben wurden zusätzlich ein wenig abgewandelt, um das Auge nicht zu sehr zu beanspruchen.\\

% Verteilung
Die relative Häufigkeit der Farben basiert auf den Gegenständen des Merkmals Industrie.
Also in diesem Fall auf den Artikeln, welche unter dem Knoten Industrie liegen.
In \autoref{fig:industry-colored} macht jede Farbe exakt $1/3$ des Knotens aus.
Die Verteilung der Farben ist also gleichmäßig.
Ein Beispiel hierfür wären neun Artikel, wovon jeweils drei Artikel sich positiv, negativ und neutral rechtfertigen.
In der Realität ist es jedoch sehr unwahrscheinlich, dass die Verteilung der Farben in einem Knoten immer gleichmäßig ist.
Viel eher wird es Verhältnisse geben, bei denen eine Farbe überwiegt und andere Farben nur in geringem Maße oder gar nicht vorkommen.\\

% Versuchsaufbau
% Adobe XD
% Prototyp Liste
% Prototyp FCA + Tutorial
% Aufgaben
% Ziele
% AttrakDiff
% QUESI
% Eigener Fragebogen
% Auswahl Testpersonen
% Transkripte
