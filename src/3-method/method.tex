\section{Methodik}
% Einleitung
In diesem Abschnitt wird das Design des Experiments vorgestellt, sowie eine kurze Beschreibung der verwendeten Technologien und Daten.
Es wird im Detail erläutert wie das Experiment aufgebaut ist und welche Ziele damit verfolgt werden.
Die konkrete Auswertung der erhobenen Daten wird im folgenden \autoref{sec:results} vorgestellt.\\

% Datensatz
Für das Experiment liegen die von \textcolor{red}{NAME} bereitgestellten Daten vor.
Es handelt sich dabei um Artikel vom Focus-Magazin\footnote{\url{https://www.focus.de}} zum Thema Elektromobilität.
Diese Artikel wurden durch \textcolor{red}{Soziologen} aufbereitet und in die, im \autoref{subsec:economics-of-conventions} erwähnten, Welten der \ac{EC} eingeteilt.
Der Datensatz besteht aus 26 Artikeln, welche in die Welten Markt, Industrie, \textcolor{red}{TODO} und Grün unterteilt wurden.
Im Datensatz ist die Welt als \textcolor{red}{TODO} benannt, weil es in der Literatur unterschiedliche Übersetzungen und Bezeichnungen für die Welten gibt.
Zusätzlich zu der Kategorisierung in eine Welt enthält der Datensatz die Rechtfertigung bezüglich der Welt.
Diese Rechtfertigung gibt an, ob der Artikel für, gegen oder beides bezüglich einer Welt ist.
Dabei bedeutet beides nicht, dass der Artikel neutral ist, sondern dass Textabschnitte im Artikel auftauchen, welche für und gegen die Welt sprechen.\\

% Vorstudie
Es wurde ebenfalls eine qualitative Vorstudie durchgeführt, welches die Grundlage für die Masterarbeit bildet.
Diese Vorstudie wurde durch \textcolor{red}{NAME} durchgeführt und \textcolor{red}{TODO: Anhang + cite Studie}.
Zweck dieser Vorstudie war es neue Darstellungsformen der Artikel-Navigation im Online-Journalismus zu erforschen.
Unter anderem hat sich in dieser Vorstudie herausgestellt, dass die Darstellung als Begriffsverband für fünf von acht Testpersonen als informativ und interessant empfunden wurde.
Allerdings gab es auch zahlreiche Kritikpunkte am Begriffsverband in der Vorstudie.
Der Begriffsverband wies für drei Personen eine hohe Komplexität auf.
Diese sorgt dafür, dass entweder eine lange Einarbeitungszeit nötig ist oder Nutzende abgeschreckt werden.
Ebenfalls ist die Unterteilung in unterschiedliche Rechtfertigung (positiv, negativ oder beides) bezüglich einer Welt für zwei Personen eher verwirrend als sinnvoll.\\

