\section{Methodik}
% Einleitung
In diesem Abschnitt wird das Design des Experiments vorgestellt, sowie eine kurze Beschreibung der verwendeten Technologien und Daten.
Es wird im Detail erläutert wie das Experiment aufgebaut ist und welche Ziele damit verfolgt werden.
Die konkrete Auswertung der erhobenen Daten wird im folgenden \autoref{sec:results} vorgestellt.\\

% Datensatz
Für das Experiment liegen die von \textcolor{red}{NAME} bereitgestellten Daten vor.
Es handelt sich dabei um Artikel vom Focus-Magazin\footnote{\url{https://www.focus.de}} zum Thema Elektromobilität.
Diese Artikel wurden durch \textcolor{red}{Soziologen} aufbereitet und in die, im \autoref{subsec:economics-of-conventions} erwähnten, Welten der \ac{EC} eingeteilt.
Der Datensatz besteht aus 26 Artikeln, welche in die Welten Markt, Industrie, \textcolor{red}{TODO} und Grün unterteilt wurden.
Im Datensatz ist die Welt als \textcolor{red}{TODO} benannt, weil es in der Literatur unterschiedliche Übersetzungen und Bezeichnungen für die Welten gibt.
Zusätzlich zu der Kategorisierung in eine Welt enthält der Datensatz die Rechtfertigung bezüglich der Welt.
Diese Rechtfertigung gibt an, ob der Artikel für, gegen oder beides bezüglich einer Welt ist.
Dabei bedeutet beides nicht, dass der Artikel neutral ist, sondern dass Textabschnitte im Artikel auftauchen, welche für und gegen die Welt sprechen.\\

% Vorstudie
Es wurde ebenfalls eine qualitative Vorstudie durchgeführt, welches die Grundlage für die Masterarbeit bildet.
Diese Vorstudie wurde durch \textcolor{red}{NAME} durchgeführt und \textcolor{red}{TODO: Anhang + cite Studie}.
Zweck dieser Vorstudie war es neue Darstellungsformen der Artikel-Navigation im Online-Journalismus zu erforschen.
Unter anderem hat sich in dieser Vorstudie herausgestellt, dass die Darstellung als Begriffsverband für fünf von acht Testpersonen als informativ und interessant empfunden wurde.
Allerdings gab es auch zahlreiche Kritikpunkte am Begriffsverband in der Vorstudie.
Der Begriffsverband wies für drei Personen eine hohe Komplexität auf.
Diese sorgt dafür, dass entweder eine lange Einarbeitungszeit nötig ist oder Nutzende abgeschreckt werden.
Ebenfalls ist die Unterteilung in unterschiedliche Rechtfertigung (positiv, negativ oder beides) bezüglich einer Welt für zwei Personen eher verwirrend als sinnvoll.\\

\subsection{Prototyp - Begriffsverband}
Im Rahmen der Masterarbeit wurde ein Prototyp entwickelt, welcher die Ergebnisse der Vorstudie aufgreift und weiterentwickelt.
Er ist somit eine weitere Iteration auf dem Weg zur finalen Umsetzung.
Dieser Abschnitt erläutert möglichst detailliert die einzelnen Schritte, welche zum Prototypen geführt haben.\\

\begin{figure}[!ht]
    \centering
    \begin{tikzpicture}
        % Erster Graph
        \begin{scope}[every node/.style={circle,thick,draw}]
            \node (A) at (0,1) {};
            \node (B) at (-2,-1) {};
            \node (C) at (2,-1) {};
            \node (D) at (0, -3) {};
        \end{scope}

        % Merkmale
        \draw (A) +(0,0.25) node[above] {Industrie};
        \draw (B) +(0,0.25) node[above, align=left] {Industrie \texttt{+}};
        \draw (C) +(0,0.25) node[above, align=left] {Industrie -};
        \draw (D) +(0,0.25) node[above, align=left] {};

        % Relationen
        \draw (A) -- (B);
        \draw (A) -- (C);
        \draw (B) -- (D);
        \draw (C) -- (D);

        % Zweiter Graph
        \begin{scope}[every node/.style={circle,thick,draw}, xshift=5cm]
            \node (A2) at (2,-1) {};
        \end{scope}

        % Merkmale
        \draw (A2) +(0,0.25) node[above] {Industrie};

        % Unsichtbare Koordinaten für den Pfeil
        \coordinate[right=1.5cm of C] (C_right);
        \coordinate[left=1.5cm of A2] (A2_left);

        % Pfeil zwischen den Graphen
        \draw[-{Stealth[length=3mm]}, thick] (C_right) -- (A2_left);
    \end{tikzpicture}
    \caption{Begriffsverband - Vorstudie (links) und vereinfachte Darstellung (rechts)}
    \label{fig:industry-comparison}
\end{figure}

% Graph
Die hohe Komplexität wurde dadurch reduziert, dass die Anzahl der Elemente im Begriffsverband reduziert wurden.
Da für zwei Personen die Unterteilung in positiv, negativ oder beides verwirrend war, wurden diese Knoten aus dem Begriffsverband entfernt.
Dies wird in \autoref{fig:industry-comparison} beispielhaft dargestellt.
Der Knoten Industrie besitzt in der Vorstudie zwei weitere Nachfolgeknoten, welche eine positive und negative Rechtfertigung für die Welt Industrie darstellen.
Der unterste Knoten vereinigt die Eigenschaften aller darüberliegenden Knoten.
Auf diese Weise gehören Artikel, welche im untersten Knoten liegen zu der Welt Industrie und rechtfertigen sich positiv und negativ bezüglich dieser Welt.\\

\begin{figure}[!ht]
    \centering
    \begin{tikzpicture}
        \definecolor{nodegreen}{HTML}{00A86B}
        \definecolor{nodeyellow}{HTML}{FFB50F}
        \definecolor{nodered}{HTML}{D32F2F}
        \begin{scope}[every node/.style={circle,thick,draw}]
            \clip (-1.1,0) rectangle (-0.34,1.1);
            \node[minimum size=2cm, fill=nodegreen] (A) at (0,0) {};
        \end{scope}
        \begin{scope}[every node/.style={circle,thick,draw}]
            \clip (-0.34,0) rectangle (0.32,1.1);
            \node[minimum size=2cm, fill=nodeyellow] (B) at (0,0) {};
        \end{scope}
        \begin{scope}[every node/.style={circle,thick,draw}]
            \clip (0.32,0) rectangle (1.1,1.1);
            \node[minimum size=2cm, fill=nodered] (C) at (0,0) {};
        \end{scope}

        \coordinate[right=0cm of A] (A_right);
        \coordinate[left=0cm of A] (A_left);
        \draw[-, thick] (A_left) -- (A_right);
    \end{tikzpicture}
    \caption{Einfärbung des Knotens Industrie}
    \label{fig:industry-colored}
\end{figure}

% Knotenfärbung
Der aus der Reduktion resultierende Graph enthält statt der vier Knoten nur noch einen Knoten.
Wenn man dies für den ganzen Graphen anwendet, dann kann der Graph \textcolor{red}{um x Knoten - nochmal nachprüfen} reduziert werden.
Der reduzierte Graph enthält jedoch auch weniger Informationen, da die Rechtfertigungen für die unterschiedlichen Welten nicht mehr dargestellt werden.
Um diesem Problem entgegenzuwirken, wurde die Repräsentation des Knotens aus \autoref{fig:industry-comparison} um drei Farben erweitert.
Diese Farben werden in \autoref{fig:industry-colored} dargestellt.
Die Farben sollen angeben, ob die Artikel sich innerhalb dieses Knotens positiv (grün), negativ (rot) oder neutral (gelb) rechtfertigen.\\

% Drei Farben
Diese drei Farben wurden ausgewählt, weil sie für die meisten Menschen auf dieselbe Weise emotional assoziiert werden.
So zeigen die Daten einer Studie, dass, aus der Menge der Grundfarbtöne, die Farbe Grün am positivsten und Rot am negativsten assoziiert wird \cite{color-emotion}.
Ebenfalls sind Grün und Rot Komplementärfarben, welche zwei Gegenpole darstellen sollen.
Die Farbe Gelb wird in der Studie als größtenteils positiv von Testpersonen wahrgenommen.
Gelb wurde ausgewählt, weil die Farbe Gelb bei Ampeln zum Beispiel als Zwischenzustand zwischen Rot und Grün dient.
Ebenfalls liegt Gelb im Farbspektrum zwischen Rot und Grün und ist ebenfalls ein Grundfarbton.
Auf diese Art und Weise kann möglicherweise eine Verknüpfung im Hirn für die Farbe Gelb als Zwischenzustand hergestellt werden.
An dieser Stelle sei angemerkt, dass Farben je nach Kontext unterschiedlich interpretiert werden können.
So kann die Farbe Rot, laut Studie, als warme und leidenschaftliche, aber auch als aggressive und intensive Farbe wahrgenommen werden.
Die Farben wurden zusätzlich ein wenig abgewandelt, um das Auge nicht zu sehr zu beanspruchen.\\

% Verteilung
Die relative Häufigkeit der Farben basiert auf den Gegenständen des Merkmals Industrie.
Also in diesem Fall auf den Artikeln, welche unter dem Knoten Industrie liegen.
In \autoref{fig:industry-colored} macht jede Farbe exakt $1/3$ des Knotens aus.
Die Verteilung der Farben ist also gleichmäßig.
Ein Beispiel hierfür wären neun Artikel, wovon jeweils drei Artikel sich positiv, negativ und neutral rechtfertigen.
In der Realität ist es jedoch sehr unwahrscheinlich, dass die Verteilung der Farben in einem Knoten immer gleichmäßig ist.
Viel eher wird es Verhältnisse geben, bei denen eine Farbe überwiegt und andere Farben nur in geringem Maße oder gar nicht vorkommen.\\

% Knotendarstellung
Es gibt viele verschiedene Möglichkeiten, wie die Knoten innerhalb eines Begriffsverbands dargestellt werden können.
In der Literatur werden die Knoten jedoch meistens nur als Kreise dargestellt, welche Beschriftungen über und unter sich tragen.
In der Vorstudie wurde eine ähnliche Darstellung verwendet.
Die Knoten wurden in zwei Hälften geteilt, wobei die obere Hälfte Welten aus den \ac{EC} und die untere Hälfte die jeweiligen Artikel repräsentiert.
Dabei können mehrere Welten und Artikel in einem Knoten liegen.
Falls der Knoten explizit zu einer oder mehreren Welten gehört, dann wird der obere Halbkreis blau eingefärbt.
Falls Artikel innerhalb eines Knotens liegen, dann wird der untere Halbkreis gelb eingefärbt. \textcolor{red}{Gibt es einen Grund für die Farben?}
Zusätzlich dazu wird am Knoten eine Zahl angegeben, welche aufsummiert angeben soll wie viele Artikel innerhalb dieses und den darunterliegenden Knoten liegen.
Diese Summe wird auch in der Literatur verwendet, jedoch tritt diese Summe nicht in Kombination mit den konkreten Gegenständen auf.
Die Gegenstände werden erst aufgelistet, sobald eine Interaktion mit einem entsprechenden Knoten erfolgt. \\

% Zeichensatz
Das Problem an dieser Darstellung ist, dass permanent alle Artikel für die Nutzenden angezeigt werden.
Dies sorgt für zusätzliches Rauschen und lenkt die Aufmerksamkeit auf die dargestellten Artikel, da diese wesentlich mehr Text ausmachen als die Welten.
Konkret werden 83 Zeichen für die Welten verwendet.
Für die Artikel werden 419 Zeichen verwendet.
Bei insgesamt 502 Zeichen machen die Artikel $83,47\%$ des Gesamttextes aus.
Um ein Bewusstsein für die eigene Filterblase zu lenken, darf der Fokus jedoch nicht auf den Artikeln liegen.
Der Fokus sollte auf den Welten liegen, da diese eine Perspektive auf ein gegebenes Thema darstellen.\\

% Neues Design
Aus diesem Grund wurden im neuen Design Knoten verwendet, welche nur Beschriftungen für die Welten und die Anzahl der Artikel enthalten.
Die Artikel werden stattdessen erst aufgelistet, wenn Nutzende mit dem Knoten interagieren.
Dieses Design hat mehrere Vorteile.
Zum einen wird der dargestellte Text für Nutzende deutlich kürzer.
Dies führt zu einer besseren Lesbarkeit und reduziert die hohe Komplexität beim ersten Betrachten des Graphen.
Statt mit einer großen Menge an Informationen konfrontiert zu werden, können Nutzende sich zunächst auf die Welten und die Verknüpfungen zwischen diesen konzentrieren.
Zum anderen wird eine bewusste Entscheidung für eine Perspektive auf ein Thema erzwungen.
Nutzende müssen sich bewusst für eine Welt entscheiden, um die Artikel dieser Welt einsehen zu können.
Dies führt zwar auch dazu, dass Nutzende mehr Klicks benötigen, um die gewünschten Artikel zu erreichen, jedoch überwiegen genannten Vorteile. \\

% Unterschiedliche Arten der Interaktion
Es gibt zwei verschiedene Wege, um mit einem Knoten zu interagieren.
Es ist möglich auf den oberen Halbkreis und auf den unteren Halbkreis zu klicken.
Beide Interaktionsmöglichkeiten existierten bereits in der Vorstudie.
Die Interaktion mit dem oberen Halbkreis führt dazu, dass alle untergeordneten Knoten farblich hervorgehoben werden.
Die Leserichtung in diesem Fall ist von oben nach unten und bedeutet, dass alle Artikel bezüglich einer Welt angezeigt werden.
Mithilfe dieser Interaktion soll es möglich sein, sich einen Überblick über die Artikel einer Welt zu verschaffen.
Eine Interaktion mit dem unteren Halbkreis hingegen sorgt dafür, dass alle übergeordneten Knoten farblich hervorgehoben werden.
Die Leserichtung ist von unten nach oben.
Die Bedeutung dieser Interaktion ist, dass alle übergeordneten Welten bezüglich eines Knotens angezeigt werden.
Interpretiert werden kann diese Interaktion als eine Art Hilfsfunktion, welche es ermöglicht anzuzeigen, welche Welten zu einem bestimmten Knoten gehören.\\

% Neue Interaktion
Im neuen Prototypen werden weiterhin die Pfade und Knoten farblich hervorgehoben, jedoch werden die Artikel nicht mehr im Graphen dargestellt.
Stattdessen werden die Artikel in einer Liste neben dem Graphen angezeigt.
Diese Liste unterscheidet sich je nach Interaktion.
Eine Interaktion mit dem oberen Halbkreis listet weiterhin alle Artikel auf, welche zur ausgewählten Welt gehören.
Da im Vergleich zur Vorstudie Informationen verloren gehen würden, wenn nur die Artikel einer Welt angezeigt werden ohne die Welten selbst, wurde die Liste um eine Art Kategorisierung erweitert.
Die Artikel innerhalb der Liste werden in Anzahl der weiteren Welten gruppiert, welche den Artikel beschreiben.
Dabei gibt die Zahl Null an, dass die aufgelisteten Artikel in der ausgewählten Welt vorkommen.
Eine Eins bedeutet, dass die Artikel in der ausgewählten Welt und einer weiteren Welt vorkommen.\\

% Beispiel Welten Liste
Ein Beispiel hilft hierbei, die Funktionsweise zu verstehen.
Angenommen es existieren drei Welten: Staat, Markt und Industrie aus den \ac{EC}.
Der erste Artikel gehört der Welt Staat an.
Der zweite Artikel gehört der Welt Staat und Markt an.
Der dritte Artikel gehört der Welt Staat und Industrie an.
Wenn nun eine Testperson auf den Knoten Staat klickt, dann werden alle Artikel aufgelistet, welche zu der Welt Staat gehören.
Bei null wäre dies nur der erste Artikel, weil nur die Welt Staat vorkommt und keine weitere Welt beschrieben wird.
Bei einer Eins wären dies der zweite und der dritte Artikel, da diese Artikel in der Welt Staat und einer weiteren Welt vorkommen.
In diesem Fall wären die zusätzlichen Welten Markt oder Industrie.\\

% Nachteil Liste
Der Nachteil an dieser Darstellung ist, dass die Liste nicht konkret aufzeigen kann, welche Artikel in welcher Welt vorkommen.
Es wäre auch möglich konkrete Welten anzugeben, jedoch würde dies die Lesbarkeit der Liste beeinträchtigen.
Das liegt daran, dass für jede zusätzliche Welt eigene Überschriften benötigt würden.
Dies würde die Liste wesentlich länger machen.
Im vorhin genannten Beispiel wären dies drei Überschriften statt zwei Überschriften.
In einem größeren Graphen mit mehr Welten wären dies wesentlich mehr.
Da diese Iterationsstufe des Prototyps darauf bedacht ist Informationen möglichst zu komprimieren, wurde auf die genaue Angabe der Welten verzichtet.
Stattdessen kann die Liste nun so gelesen werden, dass die Anzahl angibt wie weit die Artikel tatsächlich vom ausgewählten Thema entfernt sind.\\

% Interaktion unten
Eine Interaktion mit dem unteren Halbkreis führt weiterhin dazu, dass alle übergeordneten Knoten farblich hervorgehoben werden.
Da die Beschriftung der Merkmale in Form von Welten weiterhin im neuen Prototypen existieren, ist die Hilfsfunktion zur Navigation weiterhin gegeben.
Das Einzige, was wiederum wegfällt, ist die Anzeige der Artikel im Graphen.
Da eine Interaktion mit dem oberen Halbkreis bereits dazu führt, dass alle unterliegenden Artikel angezeigt werden, kam die Idee auf, dass eine Interaktion mit dem unteren Halbkreis dazu führen sollte, dass nicht erneut alle Artikel angezeigt werden.
Stattdessen beschränkt sich die Liste auf die Artikel, welche in der ausgewählten Welt vorkommen. \\

% Versuchsaufbau
% Adobe XD
% Prototyp Liste
% Prototyp FCA + Tutorial
% Aufgaben
% Ziele
% AttrakDiff
% QUESI
% Eigener Fragebogen
% Auswahl Testpersonen
% Transkripte
