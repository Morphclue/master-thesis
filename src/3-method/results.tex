\section{Evaluation}\label{sec:results}
In diesem Abschnitt werden die Ergebnisse des Experiments vorgestellt.
Dafür wird zunächst das Interview im Zusammenhang mit der Navigation durch die beiden Prototypen beschrieben.
Um einzelne Aussagen von den Teilnehmenden des Experiments zu belegen, werden Aussagen mit der Notation [P$x$, $y$] versehen.
Hierbei steht $x$ für die Nummer der Testperson und $y$ für die Zeile der Aussage im transkribierten Interviewprotokoll.
Im Anschluss werden die Auswertungen der Fragebögen vorgestellt.

\subsection{Allgemein}
% Verhalten Recherche
Zunächst werden Ergebnisse, welche sich auf allgemeine Faktoren beziehen, vorgestellt.
Dabei lässt sich festhalten, dass Personen bei ihrem Rechercheverhalten oder der Navigation unterschiedliche Ansprüche und Vorgehensweisen besitzen.
Person 1 beschreibt dabei sehr konkret ihre Vorgehensweise [P1, 28-44].
Zunächst recherchiert diese Person allgemein über ein Thema.
Anschließend werden speziellere Informationen gesucht, welche sich auf das Thema beziehen.
Zuletzt werden Themen, welche indirekt mit dem Thema in Verbindung stehen, recherchiert.
Andere Person tendieren dazu sich Artikel anzusehen, welche einen Bezug zum vorherigen Artikel haben [P3, 116-119; P6, 102-106].
Dabei scheinen diese Personen immer zwei Artikel bezüglich eines Themas zu lesen.
Person 6 begründet dies damit, dass ein Artikel meistens nicht genügt, um ein bestimmtes Thema zu verstehen.\\

%%% Fokus Titel
Durch den Aufbau des Experiments wurde ein verstärkter Fokus der Testpersonen auf die Überschriften der Artikel erzeugt.
Dies spiegelt sich bei sämtlichen Testpersonen darin wider, dass diese die Überschrift in den Vordergrund stellen.
Obwohl dies auf der einen Seite ein realistisches Verhalten abbildet, weil Testpersonen ihrem Interesse folgen, kann dies auf der anderen Seite zu einer Verzerrung der Ergebnisse führen.
Das liegt daran, dass der Fokus auf die Überschriften den Fokus auf die Titel der Artikel verlagert und nicht mehr auf die Navigation.
Vermehrt waren finanzielle oder aktuelle Themen für die Testpersonen interessant.
Person 2 erläutert zusätzlich, dass die Preissteigung von Diesel sie nicht betrifft, da sie keinen Diesel fährt [P2, 54-67].\\

% Clickbait & Werbung
Allgemein wurde das Thema Clickbait von einigen Testpersonen angesprochen [P2, 228-230; P4, 107; P4, 155-158; P5, 5-6; P10, 46-47].
Dieses Thema hat nicht direkt, sondern indirekt, etwas mit der Navigation durch die Prototypen zu tun.
Person 5 macht deutlich, dass sie diese Artikel aus Prinzip nicht klicken würde, wenn es sich um reißerische Überschriften handelt [P5, 5-6].
Person 10 hingegen erwähnt zwar, dass diese Artikel \glqq nichts aussagen\grqq{}, würde diese jedoch trotzdem lesen [P10, 46-47].
Hier wird ebenfalls deutlich, dass die Gruppe von Nutzenden keineswegs homogen ist.
Auch der Aspekt von Werbung in Artikeln wurde von diesen beiden Testpersonen angesprochen [P5, 3-4; P10, 19-21].
Das Problem Clickbait wurde bereits näher in \autoref{subsec:challenges} beschrieben.
Da die Artikel in beiden Prototypen von einer Quelle stammt, welche Klicks von Nutzenden maximieren möchten, spiegelt sich auch der Aspekt von Clickbait in den Artikeln wider.
Werbungen werden aus finanziellen Gründen in Artikeln platziert.
Sie sind einer der Gründe, weswegen die Klicks von Nutzenden als Metrik verwendet werden.

\subsection{Liste}
Während der Navigation durch die Liste ist bei den Testpersonen aufgefallen, dass sie Liste nicht intuitiv als Sortierung nach Ähnlichkeit interpretiert haben.
Auch in diesem Fall sind die Teilnehmenden nach Überschriften von Artikeln vorgegangen, um bei den Aufgaben ähnliche oder unähnliche Artikel zu finden [P2, 250-254].
Dies könnte auf unterschiedliche Arten interpretiert werden.
Zum einen könnte es daran liegen, dass das Experiment keinen ausreichenden Bezug auf andere Listensysteme mit \acp{RS} hat.
Zum anderen könnte es daran liegen, dass die Testpersonen kein allgemeines Verständnis von Listen, welche durch content-based \acp{RS} erstellt werden, haben.
Ebenfalls wurden in zwei Fällen unbekannte Überschriften als unähnlich zu einem vorgegebenen Artikel interpretiert [P3, 68-72; P4, 133-141].\\

Die in \autoref{subsec:challenges} erwähnte Problematik, dass Listen eine Art von Position Bias mit sich bringen, konnte durch eine Aussage bestätigt werden [P1, 48-53].
Person 1 erwähnte explizit, dass sie die Liste von oben nach unten liest und auch in dieser Reihenfolge konsumiert.
Eine weitere interessante Beobachtung brachte die Reihenfolge in der Personen mit den Prototypen interagieren hervor.
Die erste Aussage von Person 2 war, nachdem sie durch den Begriffsverband navigiert hatte, dass die Liste \glqq relativ zusammenhangslos\grqq{} sei und dass sie keine Form der Kategorisierung erkennt [P2, 202-203].
Sie machte zusätzlich beim Beantworten vom Fragebogen \ac{QUESI}, dass ihr diese Form der Kategorisierung in der Liste gefehlt hat [P2, 262-265].
Im gleichen Satz wurde jedoch auch erwähnt, dass Listen sehr einfach zu verwenden sind.
Dieser Gedanke wird von mehreren Testpersonen geteilt und auch von Person 3 zusammengefasst [P3, 74-78].
Listen sind für die Testpersonen einfach zugänglich und weisen eine geringere Komplexität als ein Begriffsverband auf.\\

Die einzigen beiden Personen, welche erwähnten, dass sie nur die Liste privat verwenden würden, waren Person 1 und Person 3.
Person 1 begründet dies damit, dass die Navigation durch den Begriffsverband zu umständlich ist [P1, 388-401]. 
Die Kategorisierung wird zwar als positiv empfunden, aber die Darstellungsform ist zu unübersichtlich.
Sie erwähnt explizit, dass der Gedanke nicht fallen muss: \glqq In welche Kategorie will ich mich denn jetzt begeben?\grqq{}.
In diesem Fall lässt sich jedoch auch sagen, dass das Experiment auf den ersten Blick erfolgreich war, da bei den Testpersonen ein Umdenken stattgefunden hat.
Schließlich war das Ziel des Experiments, dass Personen sich zuerst Gedanken über die Kategorie machen und sich im Anschluss Artikel aus dieser Kategorie aussuchen.
Dieses müsste jedoch noch weiter verfolgt werden, um eine konkretere Aussage treffen zu können.
Schließlich ist damit noch nicht bewiesen, dass dies auch eine Auseinandersetzung mit kritischeren Inhalten fördert.

%%% Muss sich keine Gedanken machen + Pref Recommendation
% P1, 398-400 -> done -> weitermachen bei P3
% P3, 178-184
% P3, 212-214 => Komplizierte Navigation erfordert mehr Zeit

\subsection{Begriffsverband}
%%% Darstellung interessant
% P1, 100

%%% Kategorisierung von Artikeln in Artikelansicht = Gut
% P5, 112-114
% IV+P8, 20-25
% P10, 32-36

%%% Kreis wird missverstanden
% P1, 101-103

%%% Begriff Knoten unklar
% P6, 3

%%% Umbennnung von Grün -> Umwelt
% P4, 50-52

%%% Tutorial: Aufsteigende Numierung + pref. Beispiel mit Zahlen die doppelt vorkommen
% P1, 123-138

%%% Tutorial: Letzte Seite schwierig
% P1, 181-192 -> keine Angst falsche Klicks zu setzen
% P2, 29-32
% P3, 91-92
% P4, 20-23
% P6, 28-32
% P8, 17-18
% P9, 70-71

%%% Tutorial: Letzte Seite zusätzlicher Knoten nicht notwendig
% P2, 35-41

%%% Tutorial hilfreich
% P2, 182-189
% P3, 99-101

%%% Tutorial von Hand durchgezählt
% P5, 66-67

%%% Tutorial Rindfleischburger unklar
% P7 52-54
% P9, 52-54

%%% Hovern Legende unklar -> Infopage
% P4, 25-28

%%% Unnötige Einführung von Begriff Welt
% P1, 170-172

%%% Neutral/Objektiv statt Beides
% P1, 173-178 (Auch erwähnt, dass Farben simpel sind)
% P7, 88-93 -> versteht Dinge auch als beides

%%% Überblick + Kategorien=Gut
% P1, 205-207
% P3, 193-195
% P4, 201-202
% P6, 143-147
% P9, 105-107
% P9, 150 -> direkter Vergleich zur Liste

%%% Viel positiv
% P1, 223-226
% P4, 68-70
% P5, 234-236

%%% Ebene statt +1
% P1, 237-239
% P2, 116-118 (Untergliederung in noch mehr Details)

%%% Kategorie ist der Person egal
% P1, 252-254

%%% pref. Recommendersystem
% P5, 76-78 -> bewusste Entscheidung für Kategorie
% P5, 85-88 -> Aktive Suche = Google + nicht entscheidungsfreudig

%%% Liste ist auch im Begriffsverband ein Problem
% P1, 261

%%% Unklar welche Artikel positiv / negativ
% P1, 285-287
% P6, 38-39 -> Klick auf Anteil im Knoten -> erwartet nur negative Artikel
% IV+P6, 54-60 -> Idee für Farbe

%%% Color-coding gut
% P1, 277-281 <- Erst positiv dann negativ recherche (erst sinnvoll durch color)
% P2, 27-28
% P2, 301-307
% P5, 156-160 -> intuitiv verstanden mit Anteil
% P6, 157-159 -> praktisch und übersichtlich + Kategorisierung

%%% Pref: Artikelansicht
% P2, 80-84

%%% Pref: Kategorieansicht
% P2, 94-96
% P4, 73-76

%%% Analogie FileExplorer
% P2, 122-125

%%% Aufgaben: übergeordneter Knoten
% P1, 328-332
% P2, 43-45 
% P2, 146-148
% P3, 132-135
% P4, 80-82
% P10, 56-58

%%% Aufgaben (Verständnis von Ähnlichkeit)
% P2, 162-168
% P5, 197-203

%%% Aufgaben (Verständnis von Unähnlichkeit)
% P2, 171-177
% P5, 206-208

%%% unbenannte Knoten betiteln? Design-Idee pro contra
% P2, 292-295
% P5, 176-178 -> Contra
% IV+P5, 179-182 - Designvorschlag von hovern

%%% ausgegraute Knoten weiterhin betiteln
% P5, 147-153

%%% Im Begriffsverband Texte auch highlighten
% P3, 101-102
% P5, 115-117 
% P5, 123-125 -> Designvorschlag

%%% Kategorisierung nimmt Inhalt vorweg, aber ist wie eine Summary
% P5, 134-142 + Verständnis

\subsection{AttrakDiff}
\textcolor{red}{TODO}
\subsection{QUESI}
\textcolor{red}{TODO}
\subsection{Eigener Fragebogen}
%%% YouTube News 
% P1, 373-375

%%% Kategorisierung wird als gut empfunden, aber Liste bevorzugt
% P1, 390-392
% P5, 227-228 -> Bequemlichkeit

%%% Graph in Kombination mit einer Liste
% P4, 178-183

%%% Oberbegriffsgraph vielleicht zu kompliziert
% P4, 186-187

%%% Bei Liste fehlt Zusammenhang 
% P5, 241-244

%%% Filter für Graph
% P6, 147-148

%%% Einarbeitungszeit für Graph
% P7, 127-130
