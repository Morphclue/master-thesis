\section{Evaluation}\label{sec:results}
In diesem Abschnitt werden die Ergebnisse des Experiments vorgestellt.
Dafür wird zunächst das Interview im Zusammenhang mit der Navigation durch die beiden Prototypen beschrieben.
Um einzelne Aussagen von den Teilnehmenden des Experiments zu belegen, werden Aussagen mit der Notation [P$x$, $y$] versehen.
Hierbei steht $x$ für die Nummer der Testperson und $y$ für die Zeile der Aussage im transkribierten Interviewprotokoll.
Im Anschluss werden die Auswertungen der Fragebögen vorgestellt.

\subsection{Allgemein}
%%% Verhalten Recherche
% P1, 35-43
% P1, 277-281
% P3, 116-119
% P6, 104-106 

%%% Verhalten Navigation
% P1, 181-192

%%% Fokus liegt stark auf den Titeln -> Problem der Studie

%%% Clickbait
% P2, 229-230
% P4, 107
% P4, 156-157
% P5, 5-6
% P10, 46-47

%%% Werbung -> Finanzielle Interessen
% P5, 3
% P10, 19-21

%%% Eigene Interessen Thema: Finanzen oder Aktualität
% P2, 54-67
% P7, 16
% P8, 34-36

%%% Muss sich keine Gedanken machen + Pref Recommendation
% P1, 398-400
% P3, 178-184
% P3, 212-214 => Komplizierte Navigation erfordert mehr Zeit

\subsection{Liste}
\textcolor{red}{TODO}
%%% Fokus liegt auf Artikeln + Titeln
% P1, 19-Z26

%%% Liste zusammenhangslos / keine Kategorien
% P2, 202-203
% P2, 262-265

%%% Liste von oben nach unten
% P1, 48-53 

%%% Kein Verständnis von Listenanordnung
% P2, 250-252

%%% Liste sehr leicht verständlich
% P3, 74-78

%%% Aufgabe unähnlich => Diess - Reitzle = unbekannt => unähnlich
% P3, 68-72
% P4, 138-141

\subsection{Begriffsverband}
%%% Darstellung interessant
% P1, 100

%%% Kategorisierung von Artikeln in Artikelansicht = Gut
% P5, 112-114
% IV+P8, 20-25
% P10, 32-36

%%% Kreis wird missverstanden
% P1, 101-103

%%% Begriff Knoten unklar
% P6, 3

%%% Umbennnung von Grün -> Umwelt
% P4, 50-52

%%% Tutorial: Aufsteigende Numierung + pref. Beispiel mit Zahlen die doppelt vorkommen
% P1, 123-138

%%% Tutorial: Letzte Seite schwierig
% P2, 29-32
% P3, 91-92
% P4, 20-23
% P6, 28-32
% P8, 17-18
% P9, 70-71

%%% Tutorial: Letzte Seite zusätzlicher Knoten nicht notwendig
% P2, 35-41

%%% Tutorial hilfreich
% P2, 182-189
% P3, 99-101

%%% Tutorial von Hand durchgezählt
% P5, 66-67

%%% Tutorial Rindfleischburger unklar
% P7 52-54
% P9, 52-54

%%% Hovern Legende unklar -> Infopage
% P4, 25-28

%%% Unnötige Einführung von Begriff Welt
% P1, 170-172

%%% Neutral/Objektiv statt Beides
% P1, 173-178 (Auch erwähnt, dass Farben simpel sind)
% P7, 88-93 -> versteht Dinge auch als beides

%%% Überblick + Kategorien=Gut
% P1, 205-207
% P3, 193-195
% P4, 201-202
% P6, 143-147
% P9, 105-107
% P9, 150 -> direkter Vergleich zur Liste

%%% Viel positiv
% P1, 223-226
% P4, 68-70
% P5, 234-236

%%% Ebene statt +1
% P1, 237-239
% P2, 116-118 (Untergliederung in noch mehr Details)

%%% Kategorie ist der Person egal
% P1, 252-254

%%% pref. Recommendersystem
% P5, 76-78 -> bewusste Entscheidung für Kategorie
% P5, 85-88 -> Aktive Suche = Google + nicht entscheidungsfreudig

%%% Liste ist auch im Begriffsverband ein Problem
% P1, 261

%%% Unklar welche Artikel positiv / negativ
% P1, 285-287
% P6, 38-39 -> Klick auf Anteil im Knoten -> erwartet nur negative Artikel
% IV+P6, 54-60 -> Idee für Farbe

%%% Color-coding gut
% P2, 27-28
% P2, 301-307
% P5, 156-160 -> intuitiv verstanden mit Anteil
% P6, 157-159 -> praktisch und übersichtlich + Kategorisierung

%%% Pref: Artikelansicht
% P2, 80-84

%%% Pref: Kategorieansicht
% P2, 94-96
% P4, 73-76

%%% Analogie FileExplorer
% P2, 122-125

%%% Aufgaben: übergeordneter Knoten
% P1, 328-332
% P2, 43-45 
% P2, 146-148
% P3, 132-135
% P4, 80-82
% P10, 56-58

%%% Aufgaben (Verständnis von Ähnlichkeit)
% P2, 162-168
% P5, 197-203

%%% Aufgaben (Verständnis von Unähnlichkeit)
% P2, 171-177
% P5, 206-208

%%% unbenannte Knoten betiteln? Design-Idee pro contra
% P2, 292-295
% P5, 176-178 -> Contra
% IV+P5, 179-182 - Designvorschlag von hovern

%%% ausgegraute Knoten weiterhin betiteln
% P5, 147-153

%%% Im Begriffsverband Texte auch highlighten
% P3, 101-102
% P5, 115-117 
% P5, 123-125 -> Designvorschlag

%%% Kategorisierung nimmt Inhalt vorweg, aber ist wie eine Summary
% P5, 134-142 + Verständnis

\subsection{AttrakDiff}
\textcolor{red}{TODO}
\subsection{QUESI}
\textcolor{red}{TODO}
\subsection{Eigener Fragebogen}
%%% YouTube News 
% P1, 373-375

%%% Kategorisierung wird als gut empfunden, aber Liste bevorzugt
% P1, 390-392
% P5, 227-228 -> Bequemlichkeit

%%% Graph in Kombination mit einer Liste
% P4, 178-183

%%% Oberbegriffsgraph vielleicht zu kompliziert
% P4, 186-187

%%% Bei Liste fehlt Zusammenhang 
% P5, 241-244

%%% Filter für Graph
% P6, 147-148

%%% Einarbeitungszeit für Graph
% P7, 127-130
