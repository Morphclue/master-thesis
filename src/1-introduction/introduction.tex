\section{Einleitung}
%%%% Eli Pariser
% Bei einer Suche will man nicht unbedingt möglichst diverse Ergebnisse (z.B. Supermarkt, Kiosk), bei anderen schon (z.B. Politik, Wissenschaft)
% Bilinguisten kreativer als Monolinguisten (Quellen von Charlan Nemeth und Julianne Kwan)
% 45 min. Slideshow mit anderer Kultur kann schon dabei helfen kreativer zu sein (Quelle: American Students China creativity?)
% -> Filter Bubbles können kreatives Denken einschränken
% Benkler zu Autonomie: "Um frei zu sein, muss man nicht nur die Möglichkeit haben tuen zu wollen was man will, sondern auch zu wissen was man tuen kann"
% Zu viele Optionen genau wie zu wenig = doof

%%%% Solans
% TODO: Kombination aus Ökonomie + FBA

% Selbstkontrolle = Autonomie
% wichtiger Aspekt + Freiheit erklären

% Diversität förderlich
% Boosted Kreativität

%%% Feedback
% Presse als vierte Gewalt
% Sozio-liberal vielleicht? (jedenfalls nicht komplett liberal)
% Special Issue lesen
% Experte für Kategorisierung nur grob beschreiben

%%%% Allgemein
% Zielsetzung: Liberales Modell, weil Mehrheit: Parteilichkeit und Neutralität Seite 29 \cite{reuters-2021}
% Weg mit Blackbox (why?)
% Postmaterialismus (selbstzentrierte Gesellschaft), blind für Umstände anderer Personen (explore)
% Diversifizierte Listen führen zu höherer Zufriedenheit (cite needed)
% Probleme: Coldstart, Tracking, Overfitting, Moralisch vertretbare Diversifizierung?, Große Datenmengen (cite)
% Suche: Präzise, deswegen Liste? 
