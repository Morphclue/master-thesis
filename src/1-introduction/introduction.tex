\section{Einleitung}
%%%% Pöchhacker
% Balance zwischen populären Inhalten und redaktionell interessanten Inhalten (vgl. Fürst 2017)
% Helberger 2011: Diversity by design
% Unterschiedliche Schichten wo man eingreifen könnte: Algorithmus, Daten, UI
% Algorithmus: Noise -> Zufallsinhalte -> Möglichkeit aus der Filter Bubble auszubrechen
% Algorithmus: Aufrufparameter -> Ort, Zeit, Kontext 
% "Menschen die NICHT so sind wie du, sehen diesen Beitrag"

%%%% Eli Pariser
% Bei einer Suche will man nicht unbedingt möglichst diverse Ergebnisse (z.B. Supermarkt, Kiosk), bei anderen schon (z.B. Politik, Wissenschaft)
% Bilinguisten kreativer als Monolinguisten (Quellen von Charlan Nemeth und Julianne Kwan)
% 45 min. Slideshow mit anderer Kultur kann schon dabei helfen kreativer zu sein (Quelle: American Students China creativity?)
% -> Filter Bubbles können kreatives Denken einschränken
% Besten Informationstools beschreiben wo wir stehen (z.B. in einer Bibliothek)
% Identität formt Media -> Aber was vergessen wird ist, dass Media auch Identität formt
% Benkler zu Autonomie: "Um frei zu sein, muss man nicht nur die Möglichkeit haben tuen zu wollen was man will, sondern auch zu wissen was man tuen kann"
% Zu viele Optionen genau wie zu wenig = doof

%%%% Helberger democratic role of news recommenders
% Risiko Bürgeransicht zu verengen (OFCOM 2012)
% Stereotypen und Vorurteilen Algorithmische Feedback Loop -> (quelle suchen)
% Bedenken ob Medien ohne Editoren relevante Inhalte empfehlen (Anderson 2011)
% Druck von kommerziellen Interessen (Coddington 2015) + Werbungen (Turow 2005) und viele weitere (Newman 2018, Belair-Gagnon 2018, Lewis and Usher 2016)
% Liberal RS ->
% Man kann nicht verlangen, dass Jemand sich informiert (Strömbäck 2005)
% Variation während einer Wahl wäre eine Option
% Fehlinterpretation passiert (Ekstrand and Willemsen 2016)
% RS sollten freie und autonome Entscheidungen helfen
% RS könnte customized werden
% Legenden wieso Inhalte vorgeschlagen werden
% Filterbubble aus liberaler Sicht problematisch? -> Nicht unbedingt. Hängt von der Definition ab wie ein "guter Bürger" sein sollte -> Breit gefächert oder Experte? (Li and Marsh 2008)
% Filter Bubbles werden somit zu Experten Bubbles
% Aus diesem Grund vielleicht unterschiedliche RS? Expert / General RS?
% Participatory RS ->
% RS muss viele Perspektiven abdecken
% Muss FOMO adressieren
% Muss aktive Teilnahme stimulieren und Aktivismus fördern
% Falls zuviel Zeit auf einer Perspektive verbleibt MUSS RS die Perspektive wechseln
% -> Schmaler Grad zwischen Informieren und Beeinflussen (Spahn 2012)
% Filter Bubbles sind deswegen besonders problematisch
% Da aktive Teilnahme gefördert wird könnten Funktionen aus dem Bereich der sozialen Medien relevant sein (Kommentare, Likes, etc.)
% -> Allerdings besonders kritisch zu betrachten, weil sie auch manipuliert werden können (Hassrede, Fake News, etc.)
% Pressefreiheit dient dem Volk und nicht Journalisten und deren Managern
% Deliberative RS ->
% Macht aufmerksam auf aktuelle Position und Perspektiven
% Tiefgehende Informationen
% Diversity super wichtig
% Social features (gerade bezogen auf Diskussionen)
% Rationale statt Emotionale Titel (weniger Clickbait)
% Basically ein Tool zur Bildung 

%%%% Solans
% TODO: Kombination aus Ökonomie + FBA

%%%% Allgemein
% Zielsetzung: Liberales Modell, weil Mehrheit: Parteilichkeit und Neutralität Seite 29 \cite{reuters-2021}
% Weg mit Blackbox (why?)
% Postmaterialismus (selbstzentrierte Gesellschaft), blind für Umstände anderer Personen (explore)
% Diversifizierte Listen führen zu höherer Zufriedenheit (cite needed)
% Probleme: Coldstart, Tracking, Overfitting, Moralisch vertretbare Diversifizierung?, Große Datenmengen (cite)
% Suche: Präzise, deswegen Liste? 
