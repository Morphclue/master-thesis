\section{Einleitung}
%%%% Pöchhacker
% Zersplittung von Öffentlichkeiten
% Echo Chambers: Sunstein cite
% speziell im Bezug auf den öffentlich-rechtlichen Rundfunk, aber eigentlich ein allgemeines Problem in einer Demokratie
% § 11 Abs. 2 RStV: Vielfalt der Meinungen
% Speziell in Deutschland § 5 Abs. 1 GG
% Vielfalt der Meinungen muss abgebildet werden (BVerfG, Urt. v. 11.09.2007, Rn. 115)
% § 11 Abs. 1 RStV: umfassender Überblick
% Bedenken im Urteil des BVerfG vom 11.09.2007 (RN. 118) -> Kann zur Beeinflussung von Rezipienten führen
% Balance zwischen populären Inhalten und redaktionell interessanten Inhalten (vgl. Fürst 2017)
% Mediale Grundversorgung muss im Internet gewährleistet werden (Grassmuck 2014b: 78)
% Fokus auf Popularitätsindikatoren -> einseitige infromation diet (Kulshrestha 2015)
% -> Andere Meinungen werden dadurch verschoben -> Zersplittung von öffentlichkeiten
% Holmberg 2013: Parallele Öffentlichkeiten reproduzieren sich selbst
% search: Beam 2014 Medien als Quelle für relevante Infos?
% Helberger 2011: Diversity by design
% Bewertungen werden oft zu wenig einbezogen und haben kaum Aussagekraft (Koren / Bell 2011)
% Unterschiedliche Schichten wo man eingreifen könnte: Algorithmus, Daten, UI
% Algorithmus: Noise -> Zufallsinhalte -> Möglichkeit aus der Filter Bubble auszubrechen
% Algorithmus: Aufrufparameter -> Ort, Zeit, Kontext 
% "Menschen die NICHT so sind wie du, sehen diesen Beitrag"

%%%% Eli Pariser
% Computer wie Spiegel der die eigenen Interessen reflektiert
% Suche nach Depression installiert 223 Cookies (vllt. Cookie-Gesetz einbeziehen?)
% Bereits im TV ist die Entscheidung Fox / The Nation zu schauen eine Art Filter (bezogen auf Deutschland vielleicht das Erste und RTL?) source: \cite{public-service-news}
% Eine Welt die aus bekannten Dingen besteht ist eine Welt in der es nichts zu lernen gibt
% Wenn man sein eigenes Leben bestimmen möchte muss man sich bewusst sein über die verfügbaren Optionen
% -> bei Filter Bubblen bestimmen Firmen unser selbst (bzw. der Algorithmus der Firma)
% Algorithmen sind auf KUNDEN und nicht auf Bürger zugeschnitten
% Collaborative Filtering zunächst für Spam Mails verwendet (suche nach Quelle)
% Historisch Penny Press 1830 (suche nach Quelle)
% Internet eleminiert Mittelmann (media von Latein: Mittelmann) -> News können von Nutzenden gewählt werden
% Sensationsjournalismus: Wichtige Themen sind clickbaity, sexuell, skandalös, viral
% Top Story 2005 Seattle Times Sex mit Pferd (suche nach Quelle)
% Los Angeles Times 2007: World ugliest dog (suche nach Quelle)
% Aktuellere Themen vielleicht? (suche nach Quellen)
% Push statt Pull: Nutzende suchen nicht aktiv nach Informationen, sondern werden passiv mit Informationen versorgt
% Spektrum von Sycophantic bis Parental Approach: "Was du hören willst" bis "Was du hören solltest"
% Fortschritt der Gesellschaft durch Kontakt mit Menschen die anders sind als man selbst (John Stuart Mill quote)
% Filtern ist nicht neu: Auch Tiere filtern Informationen (suche nach Quelle)
% Bezogen auf Psychologie: Schemata -> Damit wir nicht ständig neu lernen (suche nach Quelle)
% -> Studie: Frau Geburtstag -> waitress, librarian (source? sollte 1981 sein)
% -> Schemata erzeugen Dinge die nicht passiert sind (Doris Graber?)
% -> Diese werden gestärkt und es wird schwieriger neues zu lernen -> Confirmation Bias (1951 Princeton vs Dartmouth Football)
% Filter Bubble sind designed confirmation bias zu bekräftigen
% Studie zu Franz Kafka "The Country Doctor" -> Neugier (source?)
% Neugier ensteht, wenn Wissen fehlt (George Lowenstein "information gap")
% -> Muss bewusst sein, dass Wissen fehlt -> RS + Filter Bubble zeigen die Lücke nicht (versteckt), deswegen wird keine Neugier geweckt
% Bei einer Suche will man nicht unbedingt möglichst diverse Ergebnisse (z.B. Supermarkt, Kiosk), bei anderen schon (z.B. Politik, Wissenschaft)
% Bilinguisten kreativer als Monolinguisten (Quellen von Charlan Nemeth und Julianne Kwan)
% 45 min. Slideshow mit anderer Kultur kann schon dabei helfen kreativer zu sein (Quelle: American Students China creativity?)
% -> Filter Bubbles können kreatives Denken einschränken
% Entdeckung durch Serendipity - Innovative Umgebungen
% Besten Informationstools beschreiben wo wir stehen (z.B. in einer Bibliothek)
% Identität formt Media -> Aber was vergessen wird ist, dass Media auch Identität formt
% Benkler zu Autonomie: "Um frei zu sein, muss man nicht nur die Möglichkeit haben tuen zu wollen was man will, sondern auch zu wissen was man tuen kann"
% Zu viele Optionen genau wie zu wenig = doof
% Google: You are what you click
% Facebook: You are what you share
% Personalisierung erkennt den Unterschied nicht zwischen "was ich bin" und "was ich sein will", aber auch andere Probleme nicht (aufschreiben)
% -> Wir wollen fit sein, aber aktuell z.B. Süßigkeiten essen -> Filter Bubble zeigt uns nur Süßigkeiten
% Present Bias (source?)
% Kein Unterschied zwischen Impuls und generelles Interesse (clickbaits z.B.)
% Statistisch gesehen schließen Filter Bubbles Outlier aus, aber genau diese machen das Leben interessant und geben Inspiration
% Postmaterialismusm Filter Bubble maximiert Selbstoptimierung
% Demokratie funktioniert nur, wenn Altruismus da ist und Filter Bubble schubst uns in die Richtung von Egoismus
% Goldfische werden nur so groß wie ihr Aquarium -> Bei uns Menschen mit Meinungen und Werten?
% Personalisierung nur gegen Austausch von Daten und Privatsphäre
% Unterschiedliche Meinungen sind okay -> Vorschlag von Alexander: Mosaic of subcultures
% -> Erst in eingener City aufwachsen -> Andere leicht erreichbar -> Austausch? + Diversität
% Transparenz
% Vorschlag: Visitors auf Newsartikeln

%%%% Helberger democratic role of news recommenders
% Divers / nicht divers?
% Values like: Participation, transparency, deliberation and privacy?
% Alle bekommen personalisierte Inhalten denen sie zustimmen -> wofür diskussieren? -> Problem!
% Risiko Bürgeransicht zu verengen (OFCOM 2012)
% Stereotypen und Vorurteilen Algorithmische Feedback Loop -> (quelle suchen)
% Bedenken ob Medien ohne Editoren relevante Inhalte empfehlen (Anderson 2011)
% Druck von kommerziellen Interessen (Coddington 2015) + Werbungen (Turow 2005) und viele weitere (Newman 2018, Belair-Gagnon 2018, Lewis and Usher 2016)
% Welche Rolle spielen Medien in einer Demokratie? (Christians 2009, Strömbäck 2005, Dahlberg 2011, Ferree 2002, Curran 2015).
% 3 beliebteste Theorien (Karppinen 2013b): Liberal, participatory + deliberative
% Liberal Model ->
% Market place of ideas (Napoli 1999)
% Recht auf Privatsphäre und Meinungsfreiheit
% Jeder hat das Recht alles zu lesen, solange man am Ende des Tages genug informiert ist zu voten
% Benachrichtigt Communities über wichtige Themen / Krisen
% Breit informiert geht eh nicht, aber wichtigste Infos (Zaller 2003)
% Liberal RS ->
% Man kann nicht verlangen, dass Jemand sich informiert (Strömbäck 2005)
% Variation während einer Wahl wäre eine Option
% Fehlinterpretation passiert (Ekstrand and Willemsen 2016)
% RS sollten freie und autonome Entscheidungen helfen
% RS könnte customized werden
% Legenden wieso Inhalte vorgeschlagen werden
% Filterbubble aus liberaler Sicht problematisch? -> Nicht unbedingt. Hängt von der Definition ab wie ein "guter Bürger" sein sollte -> Breit gefächert oder Experte? (Li and Marsh 2008)
% Filter Bubbles werden somit zu Experten Bubbles
% Aus diesem Grund vielleicht unterschiedliche RS? Expert / General RS?
% Participatory Model ->
% Nur möglich, falls alle die Möglichkeit haben sich politisch einzusetzen
% Soziales Interesse > Egoismus
% Homo politicus (Held 2006) -> Politische Interessen sind wichtig
% Aktiver Bürger muss tiefes Wissen + politische Agenda kennen
% Medien müssen tiefgehende Informationen liefern
% Diversität muss alle signifikanten Perspektiven abdecken
% Participatory RS ->
% RS muss viele Perspektiven abdecken
% Muss FOMO adressieren
% Muss aktive Teilnahme stimulieren und Aktivismus fördern
% Falls zuviel Zeit auf einer Perspektive verbleibt MUSS RS die Perspektive wechseln
% -> Schmaler Grad zwischen Informieren und Beeinflussen (Spahn 2012)
% Filter Bubbles sind deswegen besonders problematisch
% Da aktive Teilnahme gefördert wird könnten Funktionen aus dem Bereich der sozialen Medien relevant sein (Kommentare, Likes, etc.)
% -> Allerdings besonders kritisch zu betrachten, weil sie auch manipuliert werden können (Hassrede, Fake News, etc.)
% Pressefreiheit dient dem Volk und nicht Journalisten und deren Managern
% Deliberative Model ->
% Wie Partizipativ: Soziale Interessen > Egoismus
% Diskussionen super wichtig
% Ideen vergleichen und sich mit konträren Meinungen auseinandersetzen
% Es reicht nicht aus Leute nur zu informieren
% Kontroverse Perspektiven sind die Essenz von Politik (Manin 1987)
% -> Hier können Bürger ihre Meinungen anpassen und Horizont erweitern
% Deliberative RS ->
% Macht aufmerksam auf aktuelle Position und Perspektiven
% Tiefgehende Informationen
% Diversity super wichtig
% Social features (gerade bezogen auf Diskussionen)
% Rationale statt Emotionale Titel (weniger Clickbait)
% Basically ein Tool zur Bildung 
% FAZIT:
% Es gibt kein Goldstandard
% Personalisierung im partizipativen könnte problematisch für deliberativ sein etc.
% Self-selection in einem liberalen Modell

%%%% Allgemein
% Internet > TV : \cite{reuters-2022}
% Zielsetzung: Liberales Modell, weil Mehrheit: Parteilichkeit und Neutralität Seite 29 \cite{reuters-2021}
% Weg mit Blackbox (why?)
% Postmaterialismus (selbstzentrierte Gesellschaft), blind für Umstände anderer Personen (explore)
% Diversifizierte Listen führen zu höherer Zufriedenheit (cite needed)
% Probleme: Coldstart, Tracking, Overfitting, Moralisch vertretbare Diversifizierung?, Große Datenmengen (cite)
% Probleme: Kontextabhängig: Mit Freunden schauen, privat schauen, etc.
% Suche: Präzise, deswegen Liste? 
% explizite / implizite Interaktion (cite)
% Ranking Bias (cite)
% Confirmation Bias (Reasoning about a Rule)
