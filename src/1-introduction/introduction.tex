\section{Einleitung}
Eine zunehmende Flut von Informationen im digitalen Zeitalter stellt eine Herausforderung für die Bevölkerung und technische Systeme dar.
Um diese zu bewältigen werden Systeme eingesetzt, welche die Informationsflut reduzieren und Personen dabei unterstützen, relevante Informationen zu finden.
Jedoch können solche Empfehlungen die Meinungsbildung der Nutzer beeinflussen und sie in ihrer persönlichen \glqq Filterblase\grqq{} gefangen halten.
Dieser Effekt wirkt sich negativ auf die Demokratie aus, da die Meinungsbildung der Bevölkerung nicht mehr unabhängig von den Empfehlungen der Systeme ist \cite{breaking-filter-bubble}.
Die Perspektive von Personen wird durch diese Systeme verzerrt und neue Perspektiven werden nicht mehr wahrgenommen.
Um diesen Effekt zu vermeiden, ist es wichtig, dass ein alternatives System entwickelt wird, welche diese Form der Manipulation verhindert.
Stattdessen soll das alternative System ein Bewusstsein für unterschiedliche Perspektiven schaffen und die Nutzer dazu anregen, sich mit anderen Meinungen auseinanderzusetzen. \\

Die Darstellung von Informationen bringt immer eine Form der Verzerrung der Wirklichkeit mit sich.
Diese sollte jedoch so gering wie möglich sein, um die Wirklichkeit möglichst genau abzubilden.
Häufig werden Informationen in Listen dargestellt.
Diese Form der Darstellung kann jedoch zu Problemen führen, da die Zusammenhänge zwischen den einzelnen Inhalten oft nicht ersichtlich sind.
In dieser Arbeit wird ein System vorgestellt, welches die Zusammenhänge zwischen den Inhalten besser darstellen soll.
Speziell ist das Ziel der Arbeit ein System weiterzuentwickeln, welches die Auseinandersetzung mit unterschiedlichen Inhalten fördert.
Dieses System soll die Nutzenden dazu anregen, ein breites Spektrum an Meinungen und Perspektiven wahrzunehmen.
Dadurch soll zusätzlich das persönliche Wachstum jeder einzelnen Person gefördert werden.
Denn durch die Auseinandersetzung mit unterschiedlichen Meinungen und Perspektiven können neue Erkenntnisse gewonnen werden. \\

Das System baut auf einem Prototypen auf, welcher in einer Vorstudie entwickelt wurde.
Der Prototyp und auch diese Arbeit sind Teil des Forschungsprojektes FAIRDIENSTE\footnote{FAIRDIENSTE steht für \glqq Faire digitale Dienste: Ko-Valuation in der Gestaltung datenökonomischer Geschäftsmodelle\grqq{}} \cite{fairdienste}.
Gefördert wird das Projekt vom Bundesministerium für Bildung und Forschung (BMBF).
Inhalt des Forschungsprojektes FAIRDIENSTE ist die Entwicklung datenökonomischer Geschäftsmodelle, die sowohl ethische als auch ökonomische Aspekte berücksichtigen.
Konkret bedeutet dies, dass das Interesse an Daten oft im Konflikt mit dem Interesse von Nutzenden an Privatsphäre steht.
Gerade Empfehlungssysteme sind ein Beispiel für solche Systeme, da sie Empfehlungen auf Kosten der Privatsphäre von Nutzenden erstellen.
Die Forschung umfasst soziologische und (wirtschafts-)informatische Ansätze.

\newpage

Diese Ansätze werden auch in dieser Arbeit verfolgt.
Zunächst werden im nächsten Abschnitt die Grundlagen, welche für das weitere Verständnis der Arbeit notwendig sind, erläutert.
Da Artikel aus dem Online-Journalismus ein wichtiger Bestandteil dieser Arbeit sind, wird zunächst eine historische Entwicklung der Rolle des Journalismus in der Gesellschaft dargestellt.
Zusätzlich wird die besondere Rolle des Journalismus als Stütze der Demokratie betont und daraufhin die spezielle Anforderung an ein Navigationssystem hinsichtlich Diversität und Transparenz erläutert.
Dabei wird auch auf die Rolle des Journalismus in der Demokratie eingegangen.
Im Anschluss wird die Rolle von Empfehlungssystemen in der Gesellschaft erläutert.
Da diese Systeme viele Herausforderungen mit sich bringen, werden diese ebenfalls im selben Abschnitt näher erläutert.
Schließlich werden erste Ansätze erläutert, welche den Herausforderungen entgegenwirken sollen. \\

In \autoref{subsec:economics-of-conventions} werden die Grundlagen eines Konzepts erläutert, welches aus der Soziologie stammt.
Dieses Konzept wird verwendet, um die Inhalte der Artikel in unterschiedliche Kategorien einzuordnen.
Im Anschluss darauf, in \autoref{subsec:fca}, wird ein Formalismus vorgestellt, welcher verwendet wird, um die Artikel auf hierarchische relationale Zusammenhänge zu untersuchen.
Durch die Kombination dieser beiden Ansätze entstehen neue Möglichkeiten, welche in \autoref{subsec:synergies} erläutert werden.
Die daraus resultierende Darstellungsform wird in einem Experiment untersucht.
Der Versuchsaufbau und die verwendete Methodik wird in \autoref{sec:method} vorgestellt.
Im Anschluss darauf werden die Ergebnisse des Experiments in \autoref{sec:results} präsentiert.
Anschließend werden die Ergebnisse in \autoref{sec:discussion} diskutiert.
In diesem Abschnitt werden die Grenzen der Arbeit und mögliche Erweiterungen erläutert.
In \autoref{sec:outlook} werden Ansätze erläutert, welche den entwickelten Prototypen erweitern können.
Dieser Abschnitt ist vom vorherigen Abschnitt abgegrenzt, da die hier vorgestellten Ansätze bereits zu einem gewissen Teil untersucht wurden.
Zum Abschluss wird in \autoref{sec:conclusion} ein Fazit gezogen und die Arbeit zusammengefasst.
