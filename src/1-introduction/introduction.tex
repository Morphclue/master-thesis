\section{Einleitung}
% Historisch wie sich Medien entwickelt haben (Eli Pariser)

%%%% Pöchhacker
% Zersplittung von Öffentlichkeiten
% Selektive Inhalte
% Linear -> nicht lineares Angebot
% Flut vorhandener Informationen: Aktuelle Lösung: RS
% Filter Bubbles: Eli Pariser cite
% Echo Chambers: Sunstein cite
% speziell im Bezug auf den öffentlich-rechtlichen Rundfunk, aber eigentlich ein allgemeines Problem in einer Demokratie
% § 11 Abs. 2 RStV: Vielfalt der Meinungen
% Speziell in Deutschland § 5 Abs. 1 GG
% Vielfalt der Meinungen muss abgebildet werden (BVerfG, Urt. v. 11.09.2007, Rn. 115)
% § 11 Abs. 1 RStV: umfassender Überblick
% Bedenken im Urteil des BVerfG vom 11.09.2007 (RN. 118) -> Kann zur Beeinflussung von Rezipienten führen
% Balance zwischen populären Inhalten und redaktionell interessanten Inhalten (vgl. Fürst 2017)
% Damals orienterte sich stuff am Lebensyklus von Zuschauenden (Prime Time Nachrichten, vielleicht Studie?)
% Mediale Grundversorgung muss im Internet gewährleistet werden (Grassmuck 2014b: 78)
% Fokus auf Popularitätsindikatoren -> einseitige infromation diet (Kulshrestha 2015)
% -> Andere Meinungen werden dadurch verschoben -> Zersplittung von öffentlichkeiten
% Holmberg 2013: Parallele Öffentlichkeiten reproduzieren sich selbst
% search: Beam 2014 Medien als Quelle für relevante Infos?
% Helberger 2011: Diversity by design
% content based filtering
% collaborative filtering
% neighborhood approach
% Bewertungen werden oft zu wenig einbezogen und haben kaum Aussagekraft (Koren / Bell 2011)
% Unterschiedliche Schichten wo man eingreifen könnte: Algorithmus, Daten, UI
% Algorithmus: Noise -> Zufallsinhalte -> Möglichkeit aus der Filter Bubble auszubrechen
% Algorithmus: Aufrufparameter -> Ort, Zeit, Kontext 
% "Menschen die NICHT so sind wie du, sehen diesen Beitrag"

%%%% Allgemein
% Weg mit Blackbox (why?)
% Postmaterialismus (selbstzentrierte Gesellschaft), blind für Umstände anderer Personen (explore)
% Diversifizierte Listen führen zu höherer Zufriedenheit (cite needed)
% Probleme: Coldstart, Tracking, Overfitting, Moralisch vertretbare Diversifizierung?, Große Datenmengen (cite)
% Suche: Präzise, deswegen Liste? 
% explizite / implizite Interaktion (cite)
% Ranking Bias (cite)
