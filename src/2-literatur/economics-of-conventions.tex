\subsection{Ökonomie der Konventionen} \label{subsec:economics-of-conventions}
Die Ökonomie der Konvention, auch bekannt als \ac{EC}, beschreibt verschiedene \glqq Welten\grqq{} mit diversen Werten, welche die Entscheidungen, Meinungen und Handlungen von Menschen oder Strukturen repräsentieren \cite{on-justification, oekonomie-konventionen}.
Dabei basieren die Welten auf der Annahme, dass eine gegebene Situation auf unterschiedliche Arten interpretiert und bewertet werden kann.
Diese Interpretation oder Bewertung wird als Welt bezeichnet.
Ein weiterer Hauptbestandteil der \ac{EC} ist die Rechtfertigung.
Im Rahmen der \ac{EC} wird die Rechtfertigung als Prozess definiert, welche die Entscheidungen, Meinungen und Handlungen von Menschen oder Strukturen legitimiert.
Speziell die Kritik zwingt diese zu einer Rechtfertigung.
Dabei kann die Rechtfertigung die eigene Welt bestätigen oder diese infrage stellen.
Es existieren ebenfalls Mischformen, welche die Form von Kompromissen mehrerer Welten annehmen.
\begin{center}
    \begin{table}[!ht]
        \centering
        \begin{tabular}{|l l|}
            \hline
            Welt        & Werte                               \\ \hline
            Inspiration & Spontanität, Emotion                \\
            Haus        & Tradition, Hierarchie               \\
            Markt       & Konkurrenz, Profit                  \\
            Meinung     & Bekanntheit, Anerkennung            \\
            Bürgertum   & Inklusion, Solidarität              \\
            Industrie   & Standards, Effizienz, Produktivität \\ \hline \hline
            Grün        & Umwelt                              \\
            Projekt     & Experimente, Innovation             \\ \hline
        \end{tabular}
        \caption{Acht Welten}
        \label{table:eight-worlds}
    \end{table}
\end{center}

Die unterschiedlichen Welten und ihre Werte werden in \autoref{table:eight-worlds} dargestellt.
Die Welten Inspiration, Haus, Markt, Meinung, Bürgertum und Industrie sind die sechs verschiedenen Welten, welche ursprünglich in den \ac{EC} beschrieben wurden.
Grün und Projekt sind weitere Welten, welche die ursprünglichen Welten erweitern \cite{ec-green, new-spirit-of-capitalism}.
Es sei angemerkt, dass es noch weitere Welten gibt, welche die \ac{EC} erweitern.
Diese werden in dieser Arbeit jedoch nicht betrachtet.
