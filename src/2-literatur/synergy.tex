\subsection{Synergien}\label{subsec:synergies}
In diesem Kapitel werden mögliche Vorteile aus der Kombination der \ac{FBA} und den \ac{EC}, im Zusammenhang mit News-Artikeln, aufgezeigt.
Die daraus resultierenden Synergien adressieren Probleme, welche in Zusammenhang mit einer typischen Darstellung als Liste mit dahinterliegendem \ac{RS} stehen.
Einige dieser Herausforderungen sind bereits in Kapitel \ref{subsec:challenges} aufgeführt worden.
Dabei hängen die Herausforderungen mit dem gesamten Konzept von \ac{RS} zusammen, da diese auf Basis der Daten von Nutzenden trainiert werden.
Dies führt auf der einen Seite zu einer starken Personalisierung, verletzt aber auf der anderen Seite die Privatsphäre der Nutzenden.\\

% Pöchhacker Ebenen
Die Kombination aus \ac{FBA} und \ac{EC} kann diese Probleme adressieren, da für die Darstellung von Daten im Begriffsverband keine persönlichen Daten benötigt werden.
Auch werden keine Empfehlungen für Nutzende erstellt.
Stattdessen werden die Daten in einem Begriffsverband dargestellt und die Nutzenden können sich selbst auf die Suche nach Artikeln begeben.
Dies führt zu einer starken Reduktion der Privatsphäre, da keine persönlichen Daten benötigt werden.
Dabei werden gleich zwei unterschiedliche, von Pöchhacker et al. beschriebenen, Ebenen verändert.\\

% Data traffic
Auf der einen Seite wird der Datenverkehr reduziert, da keine Daten für Inhaltsempfehlungen benötigt werden.
Dadurch fallen mehrere Probleme weg, welche mit \acp{RS} zusammenhängen.
Zum einen wird die Privatsphäre der Nutzenden gewahrt.
Zum anderen werden Metriken, welche auf Klicks basieren, irrelevant.
Die Hoffnung ist, dass dadurch die Qualität der Inhalte steigt, da die Nutzenden nicht mehr aufgrund von Klicks, sondern aufgrund von Interesse, die Inhalte auswählen.
Dies könnte dazu führen, dass der Sensationsjournalismus eingedämmt wird.
Auch Verzerrungen, wie zum Beispiel die Gegenwartsverzerrung, werden reduziert, da Nutzende den Algorithmus nicht mehr beeinflussen.
Fehlinterpretationen werden durch das Wegfallen von persönlichen Daten ebenfalls reduziert. \\

% UI
Auf der anderen Seite wird die \ac{UI} verändert, sodass Inhalte bereits aufgrund der Darstellung diverser sind.
In \autoref{subsec:first-solutions} wurden bereits einige Lösungen vorgestellt, welche bei der Darstellung von Inhalten wichtig zu beachten sind.
Wenn die Nutzenden selbst auf die Suche von Inhalten gehen müssen und keine Empfehlungen von \acp{RS} erhalten, wird das Konzept von Pull statt Push unterstützt.
Dies führt dazu, dass Nutzende sich der eigenen Blase bewusst werden können und selbst in der Kontrolle sind, welche Inhalte konsumiert werden.
Ein weiterer Vorteil der Darstellung im Begriffsverband ist, dass Nutzende die einzelnen Filterblasen auf einem Blick erkennen können.
Das liegt daran, da die Knoten im Begriffsverband die einzelnen Filterblasen repräsentieren aus der Perspektive der \ac{EC} repräsentieren.
Das Interessante an dieser Darstellung ist, dass der Begriffsverband das Konzept des Mosaiks der Subkulturen stützt. \\

% Mosaik / Neugier
Wenn Knoten im Begriffsverband Inhalte aus einer Welt der \ac{EC} repräsentieren, dann gleicht dies einer homogenen Stadt oder einer Filterblase.
Durch diese Darstellung wird deutlich, wie viele unterschiedliche Filterblasen existieren.
Dadurch, dass andere Filterblasen sichtbar werden, ist es möglich, dass die Neugier der Nutzenden geweckt wird.
Dies kann dazu führen, dass Nutzende sich in andere Filterblasen begeben und dadurch neue Perspektiven erhalten.
Diese Vorteile sind zunächst nur theoretischer Natur und einige davon werden im Experiment im folgenden Kapitel untersucht.
