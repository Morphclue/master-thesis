\section{Grundlagen}

\subsection{Geschichtlicher Hintergrund}
% Wichtig für Demokratie
Der geschichtliche Hintergrund von Nachrichten als wichtiges Medium in einer Demokratie reicht zurück bis zu den Anfängen der Demokratie in der Antike\textcolor{red}{[cite?]}.
Eine Demokratie ist einer Form der Regierung, bei der die Macht bei der Bevölkerung liegt.
Zusammenfassen lässt sich diese Form der Regierung durch Thomas Paine und Abraham Lincolns moderne Formel \glqq Regierung des Volkes, durch das Volk, für das Volk\grqq{} \cite{lincoln}.
Dies lässt sich auch erkennen, wenn man sich den Ursprung des Wortes Demokratie anschaut.
Demokratie setzt sich zusammen aus den altgriechischen Wörtern \glqq Demos\grqq{} für Volk und \glqq Kratos\grqq{} für Herrschaft. \textcolor{red}{altgriechische Schreibweise?}
Damit die Bevölkerung jedoch Herrschaft über die Regierung ausüben kann, muss diese über politische Ereignisse und Entscheidungen informiert sein.
Eine unbeeinflusste Berichterstattung ist daher eine Grundvoraussetzung für eine funktionierende Demokratie, weil nur so die Bevölkerung in der Lage ist, eine informierte Entscheidung zu treffen.

% History Penny press 1830 -> vielleicht auch bereits auf das Wort "Öffentlichkeit" eingehen?
% Gesetze speziell in Deutschland

%%%% Wie funktionieren solche Systeme aktuell?
% content based
% collaborative
% knowledge based
% hybrid

% 3 Schichten: Algorithmus, Daten, UI

%%%% Ideen aus Literatur
% Partizipativ
% Deliberal
% Liberal
% Kein Goldstandard + Erklärung wieso
% Ideen von Pöchhacker; "Menschen die NICHT so sind wie du, sehen diesen Beitrag"
