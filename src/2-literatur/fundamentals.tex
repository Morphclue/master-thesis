\section{Grundlagen}

\subsection{Geschichtlicher Hintergrund}\label{sec:story-background}
% Demokratie
Der geschichtliche Hintergrund von Nachrichten als wichtiges Medium in einer Demokratie reicht zurück bis zu den Anfängen der Demokratie in der Antike\textcolor{red}{[cite?]}.
Eine Demokratie ist einer Form der Regierung, bei der die Macht bei der Bevölkerung liegt.
Zusammenfassen lässt sich diese Form der Regierung durch Thomas Paine und Abraham Lincolns moderne Formel \glqq Regierung des Volkes, durch das Volk, für das Volk\grqq{} \cite{lincoln}.
Dies lässt sich auch erkennen, wenn man sich den Ursprung des Wortes Demokratie anschaut.
Demokratie setzt sich zusammen aus den altgriechischen Wörtern Demos (\textgreek{δῆμος}) für Volk und Kratos (\textgreek{κράτος}) für Herrschaft.
Damit die Bevölkerung jedoch Herrschaft über die Regierung ausüben kann, muss diese über politische Ereignisse und Entscheidungen informiert sein.
Eine unbeeinflusste Berichterstattung ist daher eine Grundvoraussetzung für eine funktionierende Demokratie, weil nur so die Bevölkerung in der Lage ist, eine informierte Entscheidung zu treffen. \\

% Zeitung allgemein
Obwohl der geschichtliche Hintergrund bis in die Antike zurückreicht, ist die moderne Form der Nachrichtenberichterstattung erst in den letzten Jahrhunderten entstanden.
Die \textit{Avisa Relation oder Zeitung}\footnote{Die Zeitung ist ebenfalls unter \textit{Aviso Relation oder Zeitung} auffindbar.} war die erste regelmäßig erscheinende Zeitung in Europa \cite{aviso-relation-oder-zeitung}.
Die erste Ausgabe erschien 1609 und beinhaltete Nachrichten aus verschiedenen Teilen Europas\footnote{Die erste Ausgabe trug den Untertitel: \textit{Was sich begeben vnd zugetragen hat /in Deutsch: vnd Welschland / Spannien / Niederlandt / Engellandt /Franckreich / Vngern / Osterreich / Schweden / Polen / vnnd in allen Provintzen / in Ost: vnnd West-Indien etc.}}.
Diese Nachrichten berichteten über militärische, politische und wirtschaftliche Ereignisse und Relationen zwischen den verschiedenen Staaten.
Aufgrund dieser Inhalte war die Zeitung vor allem für die Oberschicht interessant.
Diese Oberschicht bestand aus Adeligen, Diplomaten, Kaufleuten und Gelehrten.
Bis zum Ende des 18. Jahrhundert waren die meisten Zeitungen nicht für die breite Masse der Bevölkerung gedacht, sondern für eine kleine, privilegierte Gruppe.
Zu Beginn des 19. Jahrhunderts begann der Aufstieg des radikalen Journalismus, welcher sich gegen die bestehende aristokratische Ordnung richtete \cite{media-democracy}.
Erst mit der Penny Press in den 1830ern wurde die Zeitung für die breite Masse tauglich\textcolor{red}{vllt. auch cite + Amerika Urpsrung?}.
Das lag daran, dass der Inhalt dieser Zeitungen sich nun auch an Arbeiter und Immigranten richtete.
Die Penny Press war für ihren Sensationsjournalismus bekannt und berichtete über Kriminalfälle und Skandale \cite{penny-press}.\\

% Lippmann
Mit der Zeit entwickelten sich Zeitungen von einem Medium für die Oberschicht zu einem Medium für die breite Masse.
Ebenfalls wurde dadurch die Relevanz von Zeitungen für die Demokratie immer größer.
Wie bereits erwähnt, ist eine unbeeinflusste Berichterstattung eine Grundvoraussetzung für eine funktionierende Demokratie.
Diese Grundvoraussetzung wurde nach dem Ende des Ersten Weltkriegs erschüttert.
Der Journalist Walter Lippmann schrieb 1920 in seinem Buch \textit{Liberty and the News} über Propaganda und die Manipulation der Bevölkerung durch die Medien \cite{liberty-and-news}.
Konkret schrieb er:
\begin{quote}
    \glqq Wenn diejenigen, die sie kontrollieren, sich das Recht anmaßen, nach ihrem eigenen Gewissen zu bestimmen, was berichtet werden soll und zu welchem Zweck, ist die Demokratie nicht funktionsfähig\grqq{}\footnote{Sinngemäße Übersetzung}.
\end{quote}
1922 machte Lippmann erneut mit seinem Buch \textit{Public Opinion} auf die Gefahr der Manipulation durch die Medien aufmerksam \cite{public-opinion}.
Laut Lippmann muss die Bevölkerung die Fähigkeit besitzen Informationen kritisch zu hinterfragen.
Im Jahre 1925 schreibt Lippmann in seinem Buch \textit{The Phantom Public} darüber, dass die Bevölkerung nicht in der Lage ist sich umfassend mit allen politischen Fragen auseinanderzusetzen \cite{phantom-public}.
Seine Folgerung daraus ist, dass die Bevölkerung auf eine kleine Gruppe von Personen mit Expertise angewiesen ist. \\

% Dewey
Der amerikanische Philosoph John Dewey stimmte der Kritik von Lippmann zu.
Allerdings warnte Dewey vor der Gefahr, dass die Bevölkerung dadurch Informationen nur noch passiv aufnimmt.
Dewey schrieb bereits 1916 in seinem Buch \textit{Democracy and Education} \cite{democracy-and-education}:
\begin{quote}
    \glqq Eine Demokratie ist mehr als eine Regierungsform; sie ist in erster Linie eine Form des Zusammenlebens, der gemeinsamer, kommunizierten Erfahrung\grqq{}\footnote{Sinngemäße Übersetzung}.
\end{quote}
1927 betonte Dewey in seinem Buch \textit{The Public and its Problems}, dass die Bevölkerung aktiv an der politischen Debatte teilnehmen muss \cite{public-problems}.
Im Allgemeinem betont Dewey die Notwendigkeit von Bildung, um eine kritisch hinterfragende und informierte Bevölkerung zu schaffen.
Die Debatte zwischen Lippmann und Dewey war einer der größten Debatten des zwanzigsten Jahrhunderts über die Rolle von Journalismus in einer Demokratie \cite{lippmann-dewey-debate}.
Es ist schwierig ein konkretes Fazit aus dieser Debatte zu ziehen, weil beide Perspektiven ihre Berechtigung haben. \\

% Deutschland
Speziell in der Bundesrepublik Deutschland wird die Meinungsfreiheit durch das Grundgesetz (§ 5 Abs. 1 GG) geschützt \textcolor{red}{[cite?]}:
\begin{quote}
    \glqq Jeder hat das Recht, seine Meinung in Wort, Schrift und Bild frei zu äußern und zu verbreiten und sich aus allgemein zugänglichen Quellen ungehindert zu unterrichten.
    Die Pressefreiheit und die Freiheit der Berichterstattung durch Rundfunk und Film werden gewährleistet.
    Eine Zensur findet nicht statt.\grqq{}
\end{quote}
Neben den Printmedien fallen auch Rundfunk und Telemedien in Deutschland unter dieses Gesetz.
Zusätzlich werden diese Medien durch den \ac{RStV} reguliert \textcolor{red}{[cite?]}.
Der \ac{RStV} besagt, dass die öffentlich-rechtlichen Rundfunkanstalten umfassend informieren und die Meinungen der Bevölkerung wiedergeben müssen (§ 11 Abs. 1 \ac{RStV}).
Dieser Absatz macht zusätzlich aufmerksam darauf, dass die demokratischen, sozialen und kulturellen Interessen der Bevölkerung berücksichtigt werden müssen. \\

% Printmedien / Rundfunk / Telemedien
Printmedien und Rundfunk und Telemedien unterscheiden sich in vielerlei Hinsichten.
Während Rundfunk und Telemedien an eine bestimmte Zeit gebunden sind, können Printmedien jederzeit gelesen werden.
Aufgrund der zeitlichen Begrenzung müssen Inhalte sorgfältig kuriert werden.
Durch die lineare Reihenfolge der Inhalte sind bestimmte Inhalte relevanter als andere \cite{rundfunk}.
Aus diesem Grund werden die Nachrichten kurz vor dem Hauptabendprogramm\footnote{Ebenfalls \textit{Prime Time} bezeichnet} platziert.
Die Platzierung richtet sich hierbei an den Lebenszyklus des Publikums.
Der Vorteil im Vergleich zu Printmedien ist, dass Inhalte in Echtzeit veröffentlicht werden können.
Somit können Rundfunk und Telemedien auf aktuelle Ereignisse reagieren\footnote{Beispiele hierfür sind Aufklärung über Verkehrslagen im Rundfunk und sogenannte Breaking News in Telemedien}. \\

% Online-Medien
Aktuell sieht die Situation dank der Digitalisierung und der damit verbundenen Verfügbarkeit von Informationen im Internet anders aus.
Die Bevölkerung ist bei Audio- und Videoinhalten nicht mehr an eine bestimmte Zeit gebunden.
Ebenfalls können Texte und Bilder in Echtzeit veröffentlicht und verbreitet werden.
In vielen Fällen werden die traditionellen Medien ersetzt oder ergänzt.
Durch die einfache Verbreitung von Inhalten kann so ziemlich jede Person mit Internetzugang Journalismus betreiben.
Dieser Wandel führt allerdings zu einer Flut an Informationen, welche es schwer macht, relevante Inhalte zu finden.
Ein Weg diese Flut zu bewältigen ist die Nutzung von einem \ac{RS}.

\subsection{Recommender system}
% Intro RS
Ein \acf{RS} (zu Deutsch: Empfehlungssystem) ist ein Softwaresystem, welches aus einer gegebenen Menge an Daten eine Teilmenge auswählt, welche für Nutzende relevant ist \cite{recommender-systems}.
\ac{RS} werden in verschiedenen Bereichen eingesetzt.
Beispiel hierfür sind inhaltliche Empfehlungen von Artikeln in Online-Shops oder die Empfehlung von Freunden in sozialen Netzwerken \cite{ecommerce-recommender, social-recommender}.
Ein weiteres Anwendungsgebiet ist die Empfehlung von Nachrichten \cite{news-recommender}.
Vorgeschlagene Inhalte von \acp{RS} werden von vielen Personen der Bevölkerung journalistisch kurierten Empfehlungen vorgezogen \cite{recommender-preference, theory-of-machine}.
Da Menschen verschiedene Vorlieben und Interessen besitzen erscheint es logisch, dass Empfehlungen auf individueller Ebene präferiert werden.
Es existieren zahlreiche Möglichkeiten, um ein \ac{RS} zu implementieren.
Primär geht es jedoch darum die Relevanz von Inhalten für Nutzende zu bestimmen und diese zu maximieren. \\

% Überleitung Helberger
Wie bereits im vorherigen \autoref{sec:story-background} erwähnt werden Medien historisch als zentrale Informationsquelle in einem demokratischen System verwendet.
Bezogen auf das Internet kann eine Unterscheidung zwischen verteilter und demokratischer Macht\footnote{Das altgriechische Wort Kratos (\textgreek{κράτος}) kann ebenfalls als Macht übersetzt werden} vorgenommen werden \cite{free-speech-algorithmic}.
Der Unterschied liegt darin, dass bei der verteilten Macht viele Menschen betroffen werden und eine demokratische Macht viele Menschen mitentscheiden lässt.
Die Rechtswissenschaftlerin und Professorin Natali Helberger beschreibt in ihrem Paper die konkrete Beziehung zwischen \acp{RS} und diesem demokratischen System \cite{democratic-role}.
Medien spielen dabei zwei zentrale Rollen: Sie informieren und schaffen ein vielfältiges öffentliches Forum für einen Meinungsaustausch.
Dies geht Hand in Hand mit der bereits beschriebenen Notwendigkeit zu informieren und der von Dewey beschriebenen Notwendigkeit, dass die Bevölkerung aktiv an der politischen Debatte teilnimmt.
Zum Informieren werden \acp{RS} verwendet.
Dabei kann speziell bei News-\acp{RS} zwischen vier verschiedenen Algorithmen unterschieden werden (content-based, collaborative, knowledge based und hybrid).
Diese Algorithmen werden nachfolgend kurz am Beispiel einer Nachrichtenwebsite vorgestellt.\\

% Algorithmen
Content-based Filtering empfiehlt ähnliche Inhalte basierend auf Eigenschaften der Inhalte \textcolor{red}{[cite?]}.
Die Eigenschaften können beispielsweise Thema, Autor oder Zeitpunkt der Veröffentlichung sein.
In Bezug auf das Beispiel einer Nachrichtenwebsite gibt es eine Person, welche häufig Artikel über Technologie liest.
In diesem Fall würde der Algorithmus weitere Artikel zum Thema Technologie empfehlen \textcolor{red}{[cite?]}.
Collaborative Filtering empfiehlt ähnliche Inhalte basierend auf den Interessen von anderen Nutzenden.
Um die Interessen von Nutzenden zu ermitteln, muss zunächst Feedback von Nutzenden gesammelt werden.
Dieses Feedback kann explizit oder implizit gesammelt werden.
Beispiel für explizites Feedback ist das Bewerten eines Artikels.
Beispiel für implizites Feedback ist das Klicken auf einen Artikel.
Bezogen auf die Nachrichtenwebsite würde der Algorithmus Artikeln von Nutzenden mit ähnlichen Interessen empfehlen, welche positiv bewertet oder häufig geklickt wurden.
Knowledge based Filtering empfiehlt ähnliche Inhalte basierend auf explizitem Wissen über Inhalte \textcolor{red}{[cite?]}.
Beim Beispiel der Nachrichtenwebsite würde der Algorithmus Artikel empfehlen, welche sich auf dasselbe Thema beziehen oder von derselben Quelle stammen.
Es ist ebenfalls möglich mehrere der oben genannten Algorithmen zu kombinieren.
Die somit entstehende Kombination wird als ein hybrides \ac{RS} bezeichnet.
Die Kombination aus content-based und collaborative Filtering könnte somit Inhalte empfehlen, welche ähnliche Eigenschaften besitzen und von anderen Nutzenden mit ähnlichen Interessen positiv bewertet wurden.\\

% Formen der Demokratie
Helberger beschreibt ferner drei unterschiedliche Theorien der Demokratie (liberal, partizipativ und deliberativ).
Der Grund dafür ist, dass unterschiedliche Theorien unterschiedliche Anforderungen an die Medien stellen.
Die von Helberger ausgewählten Theorien sind die drei häufigsten untersuchten Theorien in den Medien- und Kommunikationswissenschaften \cite{democratic-theories}.
Zusätzlich werden die drei Theorien in Bezug auf potenzielle \ac{RS} gesetzt.
Für ein besseres Verständnis werden diese Theorien nachfolgend kurz vorgestellt.\\

% Democracy Models
Das liberale Modell der Demokratie stellt die Freiheit der Bevölkerung in den Vordergrund.
Die Bevölkerung soll die Freiheit haben, sich selbst zu informieren.
Dies umfasst auch das Recht, zu entscheiden, welche Informationen mit anderen geteilt werden und somit das Recht auf Privatsphäre.
Ähnlich wie bei der freien Marktwirtschaft wird das liberale Modell von der Annahme getragen, dass die freie Wahl und das Wettbewerbsprinzip förderlich ist.
Dieses Konzept wird metaphorisch als \glqq Marktplatz der Ideen\grqq{} bezeichnet \cite{marketplace-ideas}.
Das partizipative Modell der Demokratie hingegen stellt die Beteiligung der Bevölkerung an der politischen Debatte in den Vordergrund.
Statt sich, wie beim liberalen Modell, selbst zu informieren, soll die Bevölkerung durch die Medien informiert werden.
Inhalte müssen divers genug sein, dass sie alle signifikant relevanten Perspektiven abdecken.
Die Tendenz dieses Modells entfernt sich vom Egoismus hin zum Altruismus.
Ähnlich wie beim partizipativen Modell fokussiert sich das deliberative Modell darauf, dass das Interesse der Bevölkerung über dem eigenen Interesse steht.
Das deliberative Modell besagt jedoch, dass es nicht ausreicht, dass die Bevölkerung nur informiert wird.
Jede Person soll mit konträren Meinungen konfrontiert werden und dazu ermutigt werden sich damit auseinanderzusetzen. \\

% Democracy Models and RS
Die Theorien bzw. Modelle der Demokratie stellen unterschiedliche Anforderungen an die Medien und damit auch an ein \ac{RS}.
Diese Anforderungen können in Konflikt miteinander stehen, weswegen es unmöglich ist, ein \ac{RS} zu entwickeln, welches alle Anforderungen erfüllt.
Gerade die Anforderungen des liberalen und deliberativen Modells stehen besonders in Konflikt miteinander.
Helberger argumentiert, dass es deswegen kein Goldstandard für ein \ac{RS} gibt.
Die Masterarbeit fokussiert sich auf die Anforderungen des liberalen Modells. %TODO: Weiter ausschreiben oder woanders hin?

\subsection{Probleme von Recommender Systems}
% Intro Filterblase
Bezogen auf \acp{RS} stellt die Personalisierung von Inhalten ein Problem dar.
Diese Personalisierung von Inhalten sorgt dafür, dass die Diversität von Inhalten und andere Perspektiven eingeschränkt werden \cite{rundfunk}.
Der Effekt wurde von dem Internetaktivisten Eli Pariser im Buch \textit{The Filter Bubble: What the Internet Is Hiding from You} als Filterblase bezeichnet und ist ein bekanntes Problem in der heutigen Gesellschaft \cite{filter-bubble}.
\acp{RS} können durch die Filterblase dazu beitragen, dass Nutzende nur noch Inhalte erhalten, die sie bereits kennen.
Die empfohlenen Inhalte können dann dazu führen, dass Nutzende sich in ihrer eigenen Meinung bestätigt sehen und andere Meinungen weniger oder gar nicht wahrnehmen.
Dieser Effekt wird als Bestätigungsfehler bezeichnet und ist ebenfalls ein bekanntes Problem in der Gesellschaft \cite{reasoning-rule}. \\

% Filterwesen
Informationen werden allerdings nicht erst seit der Einführung von \acp{RS} gefiltert.
Der Mensch ist, wie alle Tiere, ein \glqq Filterwesen\grqq{} und filtert bereits auf natürliche Weise Informationen.
Die visuelle selektive Wahrnehmung ist ein Beispiel dafür \cite{selective-perception}.
Bereits in Rundfunk und Telemedien ist die Entscheidung für einen Sender eine Art Filterung, da Sender unterschiedliche Inhalte anbieten.
Das \ac{RISJ}\footnote{Das \ac{RISJ} ist eine unabhängige Stiftung an der Universität Oxford, welche sich mit Journalismus- und Medienforschung beschäftigt.} konnte in einer Studie zeigen, dass ARD, ZDF und Deutschlandradio von einer politisch linken Mehrheit und RTL und n-tv von einer politisch rechten Mehrheit konsumiert werden \cite{public-service-news}.
Dies kann an den unterschiedlichen Inhalten liegen, welche von den Sendern angeboten werden. \\

% Zurück zu RS
Bei \acp{RS} gibt es ebenfalls die Möglichkeit bewusst unterschiedliche Quellen zu wählen.
Jedoch wird aufgrund des vorliegenden Algorithmus die Diversität der Inhalte mit der Nutzungsdauer immer weiter eingeschränkt.
Wie bereits erwähnt werden \acp{RS} explizit oder implizit mit Daten gefüttert.
% TODO: Vielen ist dies nicht bewusst + QUELLE
Aufgrund der Natur von \acp{RS} werden populäre Inhalte häufiger empfohlen.
Das liegt daran, dass \acp{RS} auf der Basis von Klicks als Interaktion arbeiten.
Das Problem dabei ist, dass dadurch der Sensationsjournalismus voranschreiten und die Bevölkerung mit politisch weniger relevanten Inhalten konfrontiert werden kann.
Ebenfalls kann dies dazu führen, dass Medien nicht mehr bilden, sondern nur noch zur Unterhaltung dienen. \\

% PROBLEME?
% Expertenblase oder Filterblase
% Feedback kann bewusst / unbewusst sein
% Populäre Inhalte -> Sensationsjournalismus gewinnt -> Schlecht, um Personen zu bilden / politisch zu motivieren
% Wie werden Daten gewonnen?
% Sorge Manipulation -> BVerfG vom 11.09.2007 (RN. 118) -> Bezug auf Lippmann
% 3 Schichten: Algorithmus, Daten, UI
% Probleme auflisten mit RS
% Erste Ideen von Pöchhacker; "Menschen die NICHT so sind wie du, sehen diesen Beitrag"
