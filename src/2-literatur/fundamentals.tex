\section{Grundlagen}

\subsection{Geschichtlicher Hintergrund}
% Wichtig für Demokratie
Der geschichtliche Hintergrund von Nachrichten als wichtiges Medium in einer Demokratie reicht zurück bis zu den Anfängen der Demokratie in der Antike\textcolor{red}{[cite?]}.
Eine Demokratie ist einer Form der Regierung, bei der die Macht bei der Bevölkerung liegt.
Zusammenfassen lässt sich diese Form der Regierung durch Thomas Paine und Abraham Lincolns moderne Formel \glqq Regierung des Volkes, durch das Volk, für das Volk\grqq{} \cite{lincoln}.
Dies lässt sich auch erkennen, wenn man sich den Ursprung des Wortes Demokratie anschaut.
Demokratie setzt sich zusammen aus den altgriechischen Wörtern Demos (\textgreek{δῆμος}) für Volk und Kratos (\textgreek{κράτος}) für Herrschaft.
Damit die Bevölkerung jedoch Herrschaft über die Regierung ausüben kann, muss diese über politische Ereignisse und Entscheidungen informiert sein.
Eine unbeeinflusste Berichterstattung ist daher eine Grundvoraussetzung für eine funktionierende Demokratie, weil nur so die Bevölkerung in der Lage ist, eine informierte Entscheidung zu treffen. \\

Obwohl der geschichtliche Hintergrund bis in die Antike zurückreicht, ist die moderne Form der Nachrichtenberichterstattung erst in den letzten Jahrhunderten entstanden.
Die Avisa Relation oder Zeitung\footnote{Die Zeitung ist ebenfalls unter Aviso Relation oder Zeitung auffindbar.} war die erste regelmäßig erscheinende Zeitung in Europa\textcolor{red}{[cite?]}.
Die erste Ausgabe erschien 1609 und beinhaltete Nachrichten aus verschiedenen Teilen Europas\footnote{Die erste Ausgabe trug den Untertitel: Was sich begeben vnd zugetragen hat /in Deutsch: vnd Welschland / Spannien / Niederlandt / Engellandt /Franckreich / Vngern / Osterreich / Schweden / Polen / vnnd in allen Provintzen / in Ost: vnnd West-Indien etc.}.
Diese Nachrichten berichteten über militärische, politische und wirtschaftliche Ereignisse und Relationen zwischen den verschiedenen Staaten.
Aufgrund dieser Inhalte war die Zeitung vor allem für die Oberschicht interessant.
Diese Oberschicht bestand aus Adeligen, Diplomaten, Kaufleuten und Gelehrten.
Bis zum Ende des 18. Jahrhundert waren die meisten Zeitungen nicht für die breite Masse der Bevölkerung gedacht, sondern für eine kleine, privilegierte Gruppe.
Zu Beginn des 19. Jahrhunderts begann der Aufstieg des radikalen Journalismus, welcher sich gegen die bestehende aristokratische Ordnung richtete \cite{media-democracy}.
Erst mit der Penny Press in den 1830ern wurde die Zeitung für die breite Masse tauglich.
Das lag daran, dass der Inhalt dieser Zeitungen sich nun auch an Arbeiter und Immigranten richtete.
Die Penny Press war für ihren Sensationsjournalismus bekannt und berichtete über Kriminalfälle und Skandale \cite{penny-press}.


% History Penny press 1830 -> vielleicht auch bereits auf das Wort "Öffentlichkeit" eingehen?
% Gesetze speziell in Deutschland

%%%% Wie funktionieren solche Systeme aktuell?
% content based
% collaborative
% knowledge based
% hybrid

% 3 Schichten: Algorithmus, Daten, UI

%%%% Ideen aus Literatur
% Partizipativ
% Deliberal
% Liberal
% Kein Goldstandard + Erklärung wieso
% Ideen von Pöchhacker; "Menschen die NICHT so sind wie du, sehen diesen Beitrag"
