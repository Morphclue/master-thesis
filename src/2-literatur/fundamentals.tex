\section{Grundlagen}

\subsection{Geschichtlicher Hintergrund}\label{sec:story-background}
% Demokratie
Der geschichtliche Hintergrund von Nachrichten als wichtiges Medium in einer Demokratie reicht zurück bis zu den Anfängen der Demokratie in der Antike\textcolor{red}{[cite?]}.
Eine Demokratie ist einer Form der Regierung, bei der die Macht bei der Bevölkerung liegt.
Zusammenfassen lässt sich diese Form der Regierung durch Thomas Paine und Abraham Lincolns moderne Formel \glqq Regierung des Volkes, durch das Volk, für das Volk\grqq{} \cite{lincoln}.
Dies lässt sich auch erkennen, wenn man sich den Ursprung des Wortes Demokratie anschaut.
Demokratie setzt sich zusammen aus den altgriechischen Wörtern Demos (\textgreek{δῆμος}) für Volk und Kratos (\textgreek{κράτος}) für Herrschaft.
Damit die Bevölkerung jedoch Herrschaft über die Regierung ausüben kann, muss diese über politische Ereignisse und Entscheidungen informiert sein.
Eine unbeeinflusste Berichterstattung ist daher eine Grundvoraussetzung für eine funktionierende Demokratie, weil nur so die Bevölkerung in der Lage ist, eine informierte Entscheidung zu treffen. \\

% Zeitung allgemein
Obwohl der geschichtliche Hintergrund bis in die Antike zurückreicht, ist die moderne Form der Nachrichtenberichterstattung erst in den letzten Jahrhunderten entstanden.
Die \textit{Avisa Relation oder Zeitung}\footnote{Die Zeitung ist ebenfalls unter \textit{Aviso Relation oder Zeitung} auffindbar.} war die erste regelmäßig erscheinende Zeitung in Europa \cite{aviso-relation-oder-zeitung}.
Die erste Ausgabe erschien 1609 und beinhaltete Nachrichten aus verschiedenen Teilen Europas\footnote{Die erste Ausgabe trug den Untertitel: \textit{Was sich begeben vnd zugetragen hat /in Deutsch: vnd Welschland / Spannien / Niederlandt / Engellandt /Franckreich / Vngern / Osterreich / Schweden / Polen / vnnd in allen Provintzen / in Ost: vnnd West-Indien etc.}}.
Diese Nachrichten berichteten über militärische, politische und wirtschaftliche Ereignisse und Relationen zwischen den verschiedenen Staaten.
Aufgrund dieser Inhalte war die Zeitung vor allem für die Oberschicht interessant.
Diese Oberschicht bestand aus Adeligen, Diplomaten, Kaufleuten und Gelehrten.
Bis zum Ende des 18. Jahrhundert waren die meisten Zeitungen nicht für die breite Masse der Bevölkerung gedacht, sondern für eine kleine, privilegierte Gruppe.
Zu Beginn des 19. Jahrhunderts begann der Aufstieg des radikalen Journalismus, welcher sich gegen die bestehende aristokratische Ordnung richtete \cite{media-democracy}.
Erst mit der Penny Press in den 1830ern wurde die Zeitung für die breite Masse tauglich\textcolor{red}{vllt. auch cite + Amerika Urpsrung?}.
Das lag daran, dass der Inhalt dieser Zeitungen sich nun auch an Arbeiter und Immigranten richtete.
Die Penny Press war für ihren Sensationsjournalismus bekannt und berichtete über Kriminalfälle und Skandale \cite{penny-press}.\\

% Lippmann
Mit der Zeit entwickelten sich Zeitungen von einem Medium für die Oberschicht zu einem Medium für die breite Masse.
Ebenfalls wurde dadurch die Relevanz von Zeitungen für die Demokratie immer größer.
Wie bereits erwähnt, ist eine unbeeinflusste Berichterstattung eine Grundvoraussetzung für eine funktionierende Demokratie.
Diese Grundvoraussetzung wurde nach dem Ende des Ersten Weltkriegs erschüttert.
Der Journalist Walter Lippmann schrieb 1920 in seinem Buch \textit{Liberty and the News} über Propaganda und die Manipulation der Bevölkerung durch die Medien \cite{liberty-and-news}.
Konkret schrieb er:
\begin{quote}
    \glqq Wenn diejenigen, die sie kontrollieren, sich das Recht anmaßen, nach ihrem eigenen Gewissen zu bestimmen, was berichtet werden soll und zu welchem Zweck, ist die Demokratie nicht funktionsfähig\grqq{}\footnote{Sinngemäße Übersetzung}.
\end{quote}
1922 machte Lippmann erneut mit seinem Buch \textit{Public Opinion} auf die Gefahr der Manipulation durch die Medien aufmerksam \cite{public-opinion}.
Laut Lippmann muss die Bevölkerung die Fähigkeit besitzen Informationen kritisch zu hinterfragen.
Im Jahre 1925 schreibt Lippmann in seinem Buch \textit{The Phantom Public} darüber, dass die Bevölkerung nicht in der Lage ist sich umfassend mit allen politischen Fragen auseinanderzusetzen \cite{phantom-public}.
Seine Folgerung daraus ist, dass die Bevölkerung auf eine kleine Gruppe von Personen mit Expertise angewiesen ist. \\

% Dewey
Der amerikanische Philosoph John Dewey stimmte der Kritik von Lippmann zu.
Allerdings warnte Dewey vor der Gefahr, dass die Bevölkerung dadurch Informationen nur noch passiv aufnimmt.
Dewey schrieb bereits 1916 in seinem Buch \textit{Democracy and Education} \cite{democracy-and-education}:
\begin{quote}
    \glqq Eine Demokratie ist mehr als eine Regierungsform; sie ist in erster Linie eine Form des Zusammenlebens, der gemeinsamer, kommunizierten Erfahrung\grqq{}\footnote{Sinngemäße Übersetzung}.
\end{quote}
1927 betonte Dewey in seinem Buch \textit{The Public and its Problems}, dass die Bevölkerung aktiv an der politischen Debatte teilnehmen muss \cite{public-problems}.
Im Allgemeinem betont Dewey die Notwendigkeit von Bildung, um eine kritisch hinterfragende und informierte Bevölkerung zu schaffen.
Die Debatte zwischen Lippmann und Dewey war einer der größten Debatten des zwanzigsten Jahrhunderts über die Rolle von Journalismus in einer Demokratie \cite{lippmann-dewey-debate}.
Es ist schwierig ein konkretes Fazit aus dieser Debatte zu ziehen, weil beide Perspektiven ihre Berechtigung haben. \\

% Deutschland
Speziell in der Bundesrepublik Deutschland wird die Meinungsfreiheit durch das Grundgesetz (§ 5 Abs. 1 GG) geschützt \textcolor{red}{[cite?]}:
\begin{quote}
    \glqq Jeder hat das Recht, seine Meinung in Wort, Schrift und Bild frei zu äußern und zu verbreiten und sich aus allgemein zugänglichen Quellen ungehindert zu unterrichten.
    Die Pressefreiheit und die Freiheit der Berichterstattung durch Rundfunk und Film werden gewährleistet.
    Eine Zensur findet nicht statt.\grqq{}
\end{quote}
Neben den Printmedien fallen auch Rundfunk und Telemedien in Deutschland unter dieses Gesetz.
Zusätzlich werden diese Medien durch den \ac{RStV} reguliert \textcolor{red}{[cite?]}.
Der \ac{RStV} besagt, dass die öffentlich-rechtlichen Rundfunkanstalten umfassend informieren und die Meinungen der Bevölkerung wiedergeben müssen (§ 11 Abs. 1 \ac{RStV}).
Dieser Absatz macht zusätzlich aufmerksam darauf, dass die demokratischen, sozialen und kulturellen Interessen der Bevölkerung berücksichtigt werden müssen. \\

% Printmedien / Rundfunk / Telemedien
Printmedien und Rundfunk und Telemedien unterscheiden sich in vielerlei Hinsichten.
Während Rundfunk und Telemedien an eine bestimmte Zeit gebunden sind, können Printmedien jederzeit gelesen werden.
Aufgrund der zeitlichen Begrenzung müssen Inhalte sorgfältig kuriert werden.
Durch die lineare Reihenfolge der Inhalte sind bestimmte Inhalte relevanter als andere \cite{rundfunk}.
Aus diesem Grund werden die Nachrichten kurz vor dem Hauptabendprogramm\footnote{Ebenfalls \textit{Prime Time} bezeichnet} platziert.
Die Platzierung richtet sich hierbei an den Lebenszyklus des Publikums.
Der Vorteil im Vergleich zu Printmedien ist, dass Inhalte in Echtzeit veröffentlicht werden können.
Somit können Rundfunk und Telemedien auf aktuelle Ereignisse reagieren\footnote{Beispiele hierfür sind Aufklärung über Verkehrslagen im Rundfunk und sogenannte Breaking News in Telemedien}. \\

% Online-Medien
Aktuell sieht die Situation dank der Digitalisierung und der damit verbundenen Verfügbarkeit von Informationen im Internet anders aus.
Die Bevölkerung ist bei Audio- und Videoinhalten nicht mehr an eine bestimmte Zeit gebunden.
Ebenfalls können Texte und Bilder in Echtzeit veröffentlicht und verbreitet werden.
In vielen Fällen werden die traditionellen Medien ersetzt oder ergänzt.
Durch die einfache Verbreitung von Inhalten kann so ziemlich jede Person mit Internetzugang Journalismus betreiben.
Dieser Wandel führt allerdings zu einer Flut an Informationen, welche es schwer macht, relevante Inhalte zu finden.
Ein Weg diese Flut zu bewältigen ist die Nutzung von einem \ac{RS}.

\subsection{Recommender system}
% Intro RS
Ein \acf{RS} (zu Deutsch: Empfehlungssystem) ist ein Softwaresystem, welches aus einer gegebenen Menge an Daten eine Teilmenge auswählt, welche für Nutzende relevant ist \cite{recommender-systems}.
\ac{RS} werden in verschiedenen Bereichen eingesetzt.
Beispiel hierfür sind inhaltliche Empfehlungen von Artikeln in Online-Shops oder die Empfehlung von Freunden in sozialen Netzwerken \cite{ecommerce-recommender, social-recommender}.
Ein weiteres Anwendungsgebiet ist die Empfehlung von Nachrichten \cite{news-recommender}.
Es gibt ebenfalls zahlreiche Möglichkeiten, um ein \ac{RS} zu implementieren.
Primär geht es jedoch darum die Relevanz von Inhalten für Nutzende zu bestimmen und diese zu maximieren. \\

% Überleitung Helberger
Wie bereits im vorherigen \autoref{sec:story-background} erwähnt werden Medien historisch als zentrale Informationsquelle in einem demokratischen System verwendet.
Bezogen auf das Internet kann eine Unterscheidung zwischen verteilter und demokratischer Macht\footnote{Das altgriechische Wort Kratos (\textgreek{κράτος}) kann ebenfalls als Macht übersetzt werden} vorgenommen werden \cite{free-speech-algorithmic}.
Der Unterschied liegt darin, dass bei der verteilten Macht viele Menschen betroffen werden und eine demokratische Macht viele Menschen mitentscheiden lässt.
Die Rechtswissenschaftlerin und Professorin Natali Helberger beschreibt in ihrem Paper die konkrete Beziehung zwischen \acp{RS} und diesem demokratischen System \cite{democratic-role}.
Medien spielen dabei zwei zentrale Rollen: Sie informieren und schaffen ein vielfältiges öffentliches Forum für einen Meinungsaustausch.
Dies geht Hand in Hand mit der bereits beschriebenen Notwendigkeit zu informieren und der von Dewey beschriebenen Notwendigkeit, dass die Bevölkerung aktiv an der politischen Debatte teilnimmt.
Zum Informieren werden \acp{RS} verwendet.
Dabei kann speziell bei News-\acp{RS} zwischen vier verschiedenen Algorithmen unterschieden werden (content-based, collaborative, knowledge based und hybrid).
Diese Algorithmen werden nachfolgend kurz am Beispiel einer Nachrichtenwebsite vorgestellt.\\

% Algorithmen
Content-based Filtering empfiehlt ähnliche Inhalte basierend auf Eigenschaften der Inhalte \textcolor{red}{[cite?]}.
Die Eigenschaften können beispielsweise Thema, Autor oder Zeitpunkt der Veröffentlichung sein.
In Bezug auf das Beispiel einer Nachrichtenwebsite gibt es eine Person, welche häufig Artikel über Technologie liest.
In diesem Fall würde der Algorithmus weitere Artikel zum Thema Technologie empfehlen \textcolor{red}{[cite?]}.
Collaborative Filtering empfiehlt ähnliche Inhalte basierend auf den Interessen von anderen Nutzenden.
Um die Interessen von Nutzenden zu ermitteln, muss zunächst Feedback von Nutzenden gesammelt werden.
Dieses Feedback kann explizit oder implizit gesammelt werden.
Beispiel für explizites Feedback ist das Bewerten eines Artikels.
Beispiel für implizites Feedback ist das Klicken auf einen Artikel.
Bezogen auf die Nachrichtenwebsite würde der Algorithmus Artikeln von Nutzenden mit ähnlichen Interessen empfehlen, welche positiv bewertet oder häufig geklickt wurden.
Knowledge based Filtering empfiehlt ähnliche Inhalte basierend auf explizitem Wissen über Inhalte \textcolor{red}{[cite?]}.
Beim Beispiel der Nachrichtenwebsite würde der Algorithmus Artikel empfehlen, welche sich auf dasselbe Thema beziehen oder von derselben Quelle stammen.
Es ist ebenfalls möglich mehrere der oben genannten Algorithmen zu kombinieren.
Die somit entstehende Kombination wird als ein hybrides \ac{RS} bezeichnet.
Die Kombination aus content-based und collaborative Filtering könnte somit Inhalte empfehlen, welche ähnliche Eigenschaften besitzen und von anderen Nutzenden mit ähnlichen Interessen positiv bewertet wurden.\\

% Vorteile
% Formen der Demokratie

% Bezug auf Helberger
% Partizipativ
% Deliberal
% Liberal

% Diese Personalisierung von Inhalten sorgt allerdings ebenfalls dafür, dass die Diversität von Inhalten und andere Perspektiven eingeschränkt werden \cite{rundfunk}.
% Der Effekt wird als Filterblase bezeichnet und ist ein bekanntes Problem in der heutigen Gesellschaft \cite{filter-bubble}.
% \ac{RS} können durch die Filterblase dazu beitragen, dass Nutzende nur noch Inhalte erhalten, die sie bereits kennen.
% Die selektiven Inhalte können dann dazu führen, dass Nutzende sich in ihrer eigenen Meinung bestätigt sehen und andere Meinungen weniger oder gar nicht wahrnehmen.
% Dieser Effekt wird als Bestätigungsfehler bezeichnet und ist ebenfalls ein bekanntes Problem in der Gesellschaft \cite{reasoning-rule}.\\

% PROBLEME?
% Feedback kann bewusst / unbewusst sein
% Populäre Inhalte -> Sensationsjournalismus gewinnt -> Schlecht, um Personen zu bilden / politisch zu motivieren
% Wie werden Daten gewonnen?
% Sorge Manipulation -> BVerfG vom 11.09.2007 (RN. 118) -> Bezug auf Lippmann
% 3 Schichten: Algorithmus, Daten, UI
% Probleme auflisten mit RS
% Kein Goldstandard + Erklärung wieso
% Erste Ideen von Pöchhacker; "Menschen die NICHT so sind wie du, sehen diesen Beitrag"
