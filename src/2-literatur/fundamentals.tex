\section{Grundlagen}

\subsection{Geschichtlicher Hintergrund}
% Demokratie
Der geschichtliche Hintergrund von Nachrichten als wichtiges Medium in einer Demokratie reicht zurück bis zu den Anfängen der Demokratie in der Antike\textcolor{red}{[cite?]}.
Eine Demokratie ist einer Form der Regierung, bei der die Macht bei der Bevölkerung liegt.
Zusammenfassen lässt sich diese Form der Regierung durch Thomas Paine und Abraham Lincolns moderne Formel \glqq Regierung des Volkes, durch das Volk, für das Volk\grqq{} \cite{lincoln}.
Dies lässt sich auch erkennen, wenn man sich den Ursprung des Wortes Demokratie anschaut.
Demokratie setzt sich zusammen aus den altgriechischen Wörtern Demos (\textgreek{δῆμος}) für Volk und Kratos (\textgreek{κράτος}) für Herrschaft.
Damit die Bevölkerung jedoch Herrschaft über die Regierung ausüben kann, muss diese über politische Ereignisse und Entscheidungen informiert sein.
Eine unbeeinflusste Berichterstattung ist daher eine Grundvoraussetzung für eine funktionierende Demokratie, weil nur so die Bevölkerung in der Lage ist, eine informierte Entscheidung zu treffen. \\

% Zeitung allgemein
Obwohl der geschichtliche Hintergrund bis in die Antike zurückreicht, ist die moderne Form der Nachrichtenberichterstattung erst in den letzten Jahrhunderten entstanden.
Die \textit{Avisa Relation oder Zeitung}\footnote{Die Zeitung ist ebenfalls unter \textit{Aviso Relation oder Zeitung} auffindbar.} war die erste regelmäßig erscheinende Zeitung in Europa \cite{aviso-relation-oder-zeitung}.
Die erste Ausgabe erschien 1609 und beinhaltete Nachrichten aus verschiedenen Teilen Europas\footnote{Die erste Ausgabe trug den Untertitel: \textit{Was sich begeben vnd zugetragen hat /in Deutsch: vnd Welschland / Spannien / Niederlandt / Engellandt /Franckreich / Vngern / Osterreich / Schweden / Polen / vnnd in allen Provintzen / in Ost: vnnd West-Indien etc.}}.
Diese Nachrichten berichteten über militärische, politische und wirtschaftliche Ereignisse und Relationen zwischen den verschiedenen Staaten.
Aufgrund dieser Inhalte war die Zeitung vor allem für die Oberschicht interessant.
Diese Oberschicht bestand aus Adeligen, Diplomaten, Kaufleuten und Gelehrten.
Bis zum Ende des 18. Jahrhundert waren die meisten Zeitungen nicht für die breite Masse der Bevölkerung gedacht, sondern für eine kleine, privilegierte Gruppe.
Zu Beginn des 19. Jahrhunderts begann der Aufstieg des radikalen Journalismus, welcher sich gegen die bestehende aristokratische Ordnung richtete \cite{media-democracy}.
Erst mit der Penny Press in den 1830ern wurde die Zeitung für die breite Masse tauglich\textcolor{red}{vllt. auch cite + Amerika Urpsrung?}.
Das lag daran, dass der Inhalt dieser Zeitungen sich nun auch an Arbeiter und Immigranten richtete.
Die Penny Press war für ihren Sensationsjournalismus bekannt und berichtete über Kriminalfälle und Skandale \cite{penny-press}.\\

% Lippmann
Mit der Zeit entwickelten sich Zeitungen von einem Medium für die Oberschicht zu einem Medium für die breite Masse.
Ebenfalls wurde dadurch die Relevanz von Zeitungen für die Demokratie immer größer.
Wie bereits erwähnt, ist eine unbeeinflusste Berichterstattung eine Grundvoraussetzung für eine funktionierende Demokratie.
Diese Grundvoraussetzung wurde nach dem Ende des Ersten Weltkriegs erschüttert.
Der Journalist Walter Lippmann schrieb 1920 in seinem Buch \textit{Liberty and the News} über Propaganda und die Manipulation der Bevölkerung durch die Medien \cite{liberty-and-news}.
Konkret schrieb er:
\begin{quote}
    \glqq Wenn diejenigen, die sie kontrollieren, sich das Recht anmaßen, nach ihrem eigenen Gewissen zu bestimmen, was berichtet werden soll und zu welchem Zweck, ist die Demokratie nicht funktionsfähig\grqq{}\footnote{Sinngemäße Übersetzung}.
\end{quote}
1922 machte Lippmann erneut mit seinem Buch \textit{Public Opinion} auf die Gefahr der Manipulation durch die Medien aufmerksam \cite{public-opinion}.
Laut Lippmann muss die Bevölkerung die Fähigkeit besitzen Informationen kritisch zu hinterfragen.
Im Jahre 1925 schreibt Lippmann in seinem Buch \textit{The Phantom Public} darüber, dass die Bevölkerung nicht in der Lage ist sich umfassend mit allen politischen Fragen auseinanderzusetzen \cite{phantom-public}.
Seine Folgerung daraus ist, dass die Bevölkerung auf eine kleine Gruppe von Personen mit Expertise angewiesen ist. \\

% Dewey
Der amerikanische Philosoph John Dewey stimmte der Kritik von Lippmann zu.
Allerdings warnte Dewey vor der Gefahr, dass die Bevölkerung dadurch Informationen nur noch passiv aufnimmt.
Dewey schrieb bereits 1916 in seinem Buch \textit{Democracy and Education} \cite{democracy-and-education}:
\begin{quote}
    \glqq Eine Demokratie ist mehr als eine Regierungsform; sie ist in erster Linie eine Form des Zusammenlebens, der gemeinsamer, kommunizierten Erfahrung\grqq{}\footnote{Sinngemäße Übersetzung}.
\end{quote}
1927 betonte Dewey in seinem Buch \textit{The Public and its Problems}, dass die Bevölkerung aktiv an der politischen Debatte teilnehmen muss \cite{public-problems}.
Im Allgemeinem betont Dewey die Notwendigkeit von Bildung, um eine kritisch hinterfragende und informierte Bevölkerung zu schaffen.
Die Debatte zwischen Lippmann und Dewey war einer der größten Debatten des zwanzigsten Jahrhunderts über die Rolle von Journalismus in einer Demokratie \cite{lippmann-dewey-debate}.
Es ist schwierig ein konkretes Fazit aus dieser Debatte zu ziehen, weil beide Perspektiven ihre Berechtigung haben. \\

% Speziell in Deutschland
Speziell in der Bundesrepublik Deutschland wird die Meinungsfreiheit durch das Grundgesetz (§ 5 Abs. 1 GG) geschützt:
\begin{quote}
    \glqq Jeder hat das Recht, seine Meinung in Wort, Schrift und Bild frei zu äußern und zu verbreiten und sich aus allgemein zugänglichen Quellen ungehindert zu unterrichten.
    Die Pressefreiheit und die Freiheit der Berichterstattung durch Rundfunk und Film werden gewährleistet. 
    Eine Zensur findet nicht statt.\grqq{}
\end{quote}
Neben den Printmedien werden auch Rundfunk und Telemedien in Deutschland durch den Rundfunkstaatsvertrag (RStV) reguliert.
Dieser besagt im Besonderen, dass die öffentlich-rechtlichen Rundfunkanstalten umfassend informieren und die Meinungen der Bevölkerung wiedergeben müssen (§ 11 Abs. 1 RStV).
Der Absatz macht zusätzlich aufmerksam darauf, dass die demokratischen, sozialen und kulturellen Interessen der Bevölkerung berücksichtigt werden müssen.

% Lineare System
% Nicht lineare Systeme
% News Recommenders
% Sorge Manipulation -> BVerfG vom 11.09.2007 (RN. 118) -> Bezug auf Lippmann

%%%% Wie funktionieren solche Systeme aktuell?
% content based
% collaborative
% knowledge based
% hybrid

% 3 Schichten: Algorithmus, Daten, UI

%%%% Ideen aus Literatur
% Partizipativ
% Deliberal
% Liberal
% Kein Goldstandard + Erklärung wieso
% Ideen von Pöchhacker; "Menschen die NICHT so sind wie du, sehen diesen Beitrag"
